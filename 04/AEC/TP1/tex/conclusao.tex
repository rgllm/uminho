\section{Conclusão}\label{sec:Conclusion}

Os três sistemas de aprendizagem aqui explorados são diferentes entre eles. Enquanto que as Redes Neuronais Artificiais e a Aprendizagem por Reforço são utilizadas numa fase inicial de aprendizagem, os Algoritmos Genéticos são utilizados numa fase posterior na pesquisa por uma solução válida tendo em conta um contexto do momento.
Vemos muitas vezes sistemas em que se usa um conjunto de Redes Neuronais Artificiais com Algoritmos Genéticos para criar uma solução capaz de aprender e aplicar esse conhecimento noutros contextos.
As Redes Neuronais Artificiais assimilam-se ainda mais aos Algoritmos Genéticos na medida em que são ambos inspirados na biologia e nos sistemas biológicos. Já a Aprendizagem por Reforço é aplicado em situações em que o agente tem que interagir com o ambiente, aprendendo quais as ações que têm mais recompensas. Este tipo de comportamento é também similar ao que acontece na natureza.
Vemos assim que todos os sistemas tem origem na biologia e que apesar das suas diferenças relacionam-se de certo modo.