\chapter{Introdução}

Redes neuronais artificias apresentam-se como um sistema conexionista, fortemente inspirado nas características do sistema nervoso central do ser humano. 
Apesar das RNAs serem um modelo simplificado, a sua arquitetura extremamente interligada de unidades de processamento permite que estas sejam capazes de generalizar e adquirir conhecimento, através de um processo de aprendizagem. 

Tirando partidos das suas características, vários algoritmos de \textit{Machine Learning} recaem sobre estas estruturas de aprendizagem, devido às suas capacidades de classificação e previsão em qualquer um dos paradigmas de aprendizagem. 

Neste projeto procura-se assim recorrer às capacidades de uma RNA, criada e modelada ao contexto do problema, para realizar a classificação do nível emocional de publicações da rede social \textit{Twitter}. 

No sentido de introduzir e uniformizar os conceitos em torno das RNAs, o presente relatório fornece inicialmente um descrição teórica dos principais temas associados com a modelação de uma rede neuronal artificial. 
Posteriormente, são descritas todas as etapas desenvolvidas ao longo do projeto e as decisões tomadas no decorrer deste.

Como estrutura do presente documento: o capítulo \ref{chp:rna} apresenta uma descrição breve daquilo que atualmente se entende por uma rede neuronal artificial, focando com relevo as características das suas unidades de processamento (neurónios), as arquiteturas existentes e o algoritmo de aprendizagem de \textit{Back-Propagation}; o capítulo \ref{chp:AnaliseDados} apresenta uma descrição da fase de pré processamento dos dados, focando a eliminação de ruído, seleção da informação útil e criação de novos atributos.

De seguida, numa vertente mais prática, os capítulos \ref{chp:criacaoRNAs} e \ref{cht:analiseresultados} apresentam, respetivamente, todas as considerações no processo de desenvolvimento e análise de resultados, ao longo dos diferentes cenários de teste realizados. 

Por fim, o capítulo \ref{chp:conclusao} apresenta as conclusões relativamente ao projeto e o trabalho futuro. 

\vspace{5cm}



