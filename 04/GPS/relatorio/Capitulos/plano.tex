\chapter{Plano de Trabalho}
\label{cht:planodetrabalho}


\section{Motivação}

As metodologias Ágil são no atual ambiente de desenvolvimento de software o principal meio de organização das equipas e dos planos de trabalho. Apesar de existirem no mercado muitas metodologias, as suas características e adequação ao ritmo de trabalho atual tornam as metodologias Ágil as mais apropriadas. Torna-se por isso importante perceber quando podemos considerar um projeto como realmente ágil, aplicando para isso um método de diagnóstico.

\section{Objetivos}

Visando contribuir com a base de conhecimento para preencher a lacuna da falta de um método eficaz de identificação de um projeto ágil, o presente trabalho tem como objetivos:

\begin{itemize}
    \item Identificação de um projeto Ágil;
    \item Desenvolvimento de um método eficaz de diagnóstico;
    \item Identificação se uma estratégia Ágil é a mais adequada para determinado projeto;
    \item Identificação das características de um projeto Ágil;
\end{itemize}

\section{Plano de Desenvolvimento}

Como forma de delinear os próximos passos para a concretização deste projeto, definimos as seguintes metas que iremos procurar cumprir dentro dos prazos estabelecidos:


\begin{itemize}
    \item \textbf{22 de Abril - 29 de Abril} - Investigar e discutir maneiras de responder à pergunta "Quando pode ser considerado um projeto, como realmente ágil?";
    
    \item \textbf{30 de Abril - 5 de Maio} - Exprimir no relatório as conclusões retiradas da investigação feita;
    
    \item \textbf{5 de Maio - 23 de Maio} - Desenvolver uma ferramenta que permita identificar se um determinado projeto deve seguir uma metodologia ágil ou não; documentar essa mesma ferramenta e descrever as decisões tomadas durante sua conceção. Por enquanto palpita-se que essa ferramenta será algo parecido com um formulário, mas isso será também decidido durante este período em que já teremos uma melhor perceção do domínio em que operaremos;
    
    \item \textbf{23 de Maio - 31 de Maio} - Aplicar a ferramenta desenvolvida a um projeto académico (ainda por decidir), analisar o resultado e exprimir no relatório comentários e opiniões sobre o resultado obtido;
    
    \item \textbf{31 de Maio - 9 de Junho} - Rever todo o conteúdo produzido, melhorar no caso de ser detetada alguma possível melhoria e relatar as conclusões gerais sobre a globalidade do projeto.
\end{itemize}


