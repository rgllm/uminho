%
% Layout retirado de http://www.di.uminho.pt/~prh/curplc09.html#notas
%
\documentclass{report}
%encoding
%--------------------------------------
\usepackage[utf8]{inputenc}
\usepackage[T1]{fontenc}
%--------------------------------------

%Portuguese-specific commands
%--------------------------------------
\usepackage[portuguese]{babel}
%--------------------------------------

%Hyphenation rules
%--------------------------------------
\usepackage{hyphenat}
\hyphenation{mate-mática recu-perar}
%--------------------------------------

\usepackage{url}
\usepackage{enumerate}
\usepackage{graphicx}

%\usepackage{alltt}
%\usepackage{fancyvrb}
\usepackage{listings}
%LISTING - GENERAL
\lstset{
    basicstyle=\small,
    numbers=left,
    numberstyle=\tiny,
    numbersep=5pt,
    breaklines=true,
    frame=tB,
    mathescape=true,
    escapeinside={(*@}{@*)}
}
%
%\lstset{ %
%   language=Java,                          % choose the language of the code
%   basicstyle=\ttfamily\footnotesize,      % the size of the fonts that are used for the code
%   keywordstyle=\bfseries,                 % set the keyword style
%   %numbers=left,                          % where to put the line-numbers
%   numberstyle=\scriptsize,                % the size of the fonts that are used for the line-numbers
%   stepnumber=2,                           % the step between two line-numbers. If it's 1 each line
%                                           % will be numbered
%   numbersep=5pt,                          % how far the line-numbers are from the code
%   backgroundcolor=\color{white},          % choose the background color. You must add \usepackage{color}
%   showspaces=false,                       % show spaces adding particular underscores
%   showstringspaces=false,                 % underline spaces within strings
%   showtabs=false,                         % show tabs within strings adding particular underscores
%   frame=none,                             % adds a frame around the code
%   %abovecaptionskip=-.8em,
%   %belowcaptionskip=.7em,
%   tabsize=2,                              % sets default tabsize to 2 spaces
%   captionpos=b,                           % sets the caption-position to bottom
%   breaklines=true,                        % sets automatic line breaking
%   breakatwhitespace=false,                % sets if automatic breaks should only happen at whitespace
%   title=\lstname,                         % show the filename of files included with \lstinputlisting;
%                                           % also try caption instead of title
%   escapeinside={\%*}{*)},                 % if you want to add a comment within your code
%   morekeywords={*,...}                    % if you want to add more keywords to the set
%}

\usepackage{xspace}

\parindent=0pt
\parskip=2pt

\setlength{\oddsidemargin}{-1cm}
\setlength{\textwidth}{18cm}
\setlength{\headsep}{-1cm}
\setlength{\textheight}{23cm}

\def\darius{\textsf{Darius}\xspace}
\def\antlr{\texttt{AnTLR}\xspace}
\def\pl{\emph{Processamento de Linguagens}\xspace}

\def\titulo#1{\section{#1}}
\def\super#1{{\em Supervisor: #1}\\ }
\def\area#1{{\em \'{A}rea: #1}\\[0.2cm]}
\def\resumo{\underline{Resumo}:\\ }


%%%%\input{LPgeneralDefintions}

\title{Processamento de Linguagens (3º ano do MiEI)\\ \textbf{Trabalho Prático 1 - Parte B}\\ Flex - Processador de Inglês corrente}
\author{Gustavo Andrez\\ (A27748) \and Rogério Moreira\\ (A74634) \and Samuel Ferreira\\ (A76507) }
\date{\today}

\begin{document}

\maketitle

\begin{abstract}

A diversidade e quantidade de informação nos dias de hoje é cada vez maior e mais dispersa, está em todo o lado em grandes quantidades, a todo o momento. Posto isto, torna-se ainda mais necessária uma linguagem de programação como o C Flex, que permite filtrar, de maneira facilitada, a informação essencial num ficheiro em que os dados estejam dispersos e ainda tratar essa informação. O trabalho apresentado na unidade curricular de Processamento de Linguagens tem como objetivo usar esta ferramenta para filtrar um conjunto de dados fornecidos, extraindo desses dados informação relevante.

\end{abstract}

\tableofcontents

\chapter{Introdu\c{c}\~ao} \label{intro}

\section*{Introdu\c{c}\~ao} \

A diversidade e quantidade de informação nos dias de hoje é cada vez maior e mais dispersa, está em todo o lado em grandes quantidades, a todo o momento. Posto isto, torna-se ainda mais necessária uma linguagem de programação como o C Flex, que permite filtrar, de maneira facilitada, a informação essencial num ficheiro em que os dados estejam dispersos e ainda tratar essa informação. O trabalho apresentado na unidade curricular de Processamento de Linguagens tem como objetivo usar esta ferramenta para filtrar um conjunto de dados fornecidos, extraindo desses dados informação relevante.

\section*{Sele\c{c}\~ao de enunciados} \

O tema escolhido pelo grupo foi “Processador de Inglês corrente” cuja descrição apresentamos em seguida.
\\
“Neste projeto pretende-se que leiam um texto corrente em inglês que faça uso das habituais contrações do tipo I'm, I'll, We're, can't, it's e:
\begin{description}
\item[] reproduza o texto de entrada na saída expandindo todas as contrações que encontrar.
\item[] gere em HTML uma lista ordenada de todos os verbos (no infinitivo não flexionado) que encontrar no texto, recordando que essa forma verbal em inglês precedida pela palavra to, sendo a mesma palavra retirada se o dito verbo for antecedido por can, could, shall, should, will, would, may, might. Considere também a forma interrogativa em que o verbo no infinitivo é precedido por do/does ou did.
\end{description}

\section*{Objetivos}
Para o desenvolvimento deste trabalho foram definidos os seguintes objetivos:
\\
\begin{description}
\item[] Conseguir expandir corretamente um número elevado de contrações presentes num texto em língua inglesa;
\item[] Identificar, com uma taxa de correção elevada, os verbos no infinitivo constantes num texto em língua inglesa.
\end{description}

\section*{Estrutura do Relatório} \

Inicialmente vai ser explicado o problema, as características dos dados fornecidos e os padrões de frase por nós encontrados e que levaram à resolução dos problemas inicialmente propostos.
Numa segunda fase apresentamos a codificação em Flex da solução desenvolvida mostrando as estruturas de dados e ações semânticas, bem como alternativas e decisões tomadas face aos problemas de implementação.
Em seguida mostramos exemplo de um input e output e, posteriormente, é feita uma análise de resultado onde se explicam a realização de alguns testes para caracterizar a solução encontrada.
Em apêndice a este relatório está também todo o código desenvolvido.

\section*{Características dos Dados, Padr\~oes de Frase e Decisões}

O programa a desenvolver tem por propósito receber textos em língua inglesa. Neste idioma é comum a utilização de contrações. Assim, quando estas forem detectadas far-se-á a respectiva expansão. Para isto foi necessário investigar sobre as contrações possíveis e decidir acerca de quais os casos a serem cobertos. De uma forma geral, decidiu-se cobrir todos os casos mais comuns de contrações tanto específicos (p.ex. gonna) como gerais (p.ex. we’ll).
Relativamente à recolha de verbos no infinitivo, para os localizar, considerou-se que se encontram precedidos por palavras características (p.ex. to e may). No entanto, nos casos em que a frase se encontra na forma interrogativa, entre as palavras atrás referidas e o verbo, usualmente é encontrado um pronome. Assim, assumiu-se que é um verbo no infinito a palavra que é precedida de um pronome e uma das palavras atrás referidas e seguido por um ponto de interrogação.

\chapter{Implementação e Ações Semânticas} \label{ae}
\section{Expansão de contrações}
Em seguida são discriminadas as expressões irregulares responsáveis por expandir as contrações detectadas bem como alguns exemplos abrangidos pelas mesmas.
\\
\begin{verbatim}
[gG]onna[\.\?\!\;,\ \n\t] {printf("%coing to%c",yytext[0],yytext[strlen(yytext)-1]);}
\end{verbatim}
Gonna -> Going to      e 	gonna -> going to
\\

\section{Recolha de verbos}

\section{Apresentação do Output}

\chapter{Apresentação de Resultados}
Em seguida apresentamos um pequeno exemplo de input por forma a demonstrar as funcionalidades do programa desenvolvido.
\textbf{\underline{Input - Exemplo}}
\begin{verbatim}
   It's my life
   It's now or never
   I ain't gonna live forever
   I just want to live while I'm alive
   (It's my life)
   My heart is like an open highway
   Like Frankie said
   I did it my way
   I just wanna live while I'm alive
   It's my life
\end{verbatim}
\\
\textbf{\underline{Output - Expansão de contrações}}
\begin{verbatim}
   It [is/has] my life
   It [is/has] now or never
   I am not going to live forever
   I just want to live while I am alive
   (It [is/has] my life)
   My heart is like an open highway
   Like Frankie said
   I did it my way
   I just want to live while I am alive
   It [is/has] my life
\end{verbatim}
\\
\textbf{\underline{Output - Recolha de verbos}}

\chapter{Análise de Resultados}
Perante os resultados obtidos procedemos a uma análise onde em seguida se sintetizam alguns comentários relativos à expansão de contrações e à recolhas de verbos no infinitivo.
\\
\textbf{\underline{Expansão de contrações}}
\\
\begin{itemize}
      \item
\end{itemize}
\\
\textbf{\underline{Recolha de verbos}}
\\
\begin{itemize}
      \item
\end{itemize}
\\
Obtivemos os seguintes resultados para o livro Eragon:
\begin{table}[]
\centering
\caption{My caption}
\label{my-label}
\begin{tabular}{llll}
Total & Certos & Errados & Sucesso \\
873   & 713    & 160     & 81,67
\end{tabular}
\end{table}
Resultados obtidos sem a deteção dos verbos no infinitivo nas frases interrogativas.
\\
\begin{table}[]
\centering
\caption{My caption}
\label{my-label}
\begin{tabular}{llll}
Total & Certos & Errados & Sucesso \\
876   & 715    & 161     & 81,62
\end{tabular}
\end{table}
Resultados obtidos com a deteção dos verbos no infinitivo nas frases interrogativas, o verbo tem que ser obrigatoriamente a última palavra da frase.
\\
\begin{table}[]
\centering
\caption{My caption}
\label{my-label}
\begin{tabular}{llll}
Total & Certos & Errados & Sucesso \\
876   & 714    & 162     & 81,50
\end{tabular}
\end{table}
Resultados obtidos com a deteção dos verbos no infinitivo nas frases interrogativas mas em que o verbo poderá ser a última ou penúltima palavra da frases.
\\
Tendo em conta os resultados obtidos e visto que, a taxa de sucesso e idêntica nos três casos optámos por dar privilégio ao número de verbos obtidos, tendo sido incluído na versão final o passo INTERROG.

\chapter{Conclusão} \label{concl}
Atendendo aos resultados obtidos, consideramos que todos os objectivos foram atingidos com sucesso.
Consideramos que o programa desenvolvido pode ser usado como ferramenta de tratamento de texto em diversos contextos.
Como trabalho futuro sugerimos a inclusão de mais alguns casos de contrações e a automatização da comparação dos verbos recolhidos com uma lista de verbos de modo a eliminar os casos não válidos.

\appendix
\chapter{Código GAWK}

Lista-se a seguir o código desenvolvido.
\\
\begin{verbatim}
   %option noyywrap
   %x VERBO INTERROG
   %{
   	#include "string.h"
   	#include <stdlib.h>
   	#define MAX 3000
   	char * verbos[MAX];
   	int countVerbos=0;
   	static int compare (const void * a, const void * b){
       	return strcmp (*(const char **) a, *(const char **) b);
   	}
   %}
   %%
   [^a-zA-Z]+(?i:(to|can('t)?|could|shall|should|will|would|may|might|did(n't)))\ +  { ECHO; BEGIN VERBO; }
   <VERBO>(?i:(i|me|my(self)?|mine|you(r|rs|rself)?|he|him(self)?|his|she|her(s|self)?|it(s)?|we|us|our(s|selves)?|the(y)|them(selves)?|a|th(is|at)))[\.\?\!\;\',\:\ \n\t] { ECHO ; BEGIN INTERROG;}
   <VERBO>[a-z]+ {
   	int i=0;
   	while(i<countVerbos){
       	if(strcmp(verbos[i],yytext)==0)
           	break;
       	i++;
   	}
   	if(i==countVerbos)
       	verbos[countVerbos++]=strdup(yytext);
   	ECHO;
   	BEGIN INITIAL;
   }
   <VERBO>.|\n                	{ ECHO ; BEGIN INITIAL ; }
   <INTERROG>[a-z]+(\ [a-z]+)?\? {
   	int i=0;
   	yytext[strlen(yytext)-1]='\0';
   	char * aux=strtok(yytext," ");
   	aux=strtok(aux,"?");
   	while(i<countVerbos){
       	if(strcmp(verbos[i],aux)==0)
           	break;
       	i++;
   	}
   	if(i==countVerbos)
       	verbos[countVerbos++]=strdup(aux);
   	ECHO;
   	BEGIN INITIAL;
   }
   <INTERROG>.|\n                	{ ECHO ; BEGIN INITIAL ; }



   [gG]onna[\.\?\!\;,\ \n\t]                	{ printf("%coing to%c",yytext[0],yytext[strlen(yytext)-1]); }
   [wW]anna[\.\?\!\;,\ \n\t]                	{ printf("%cant to%c",yytext[0],yytext[strlen(yytext)-1]); }
   [cC]an't[\.\?\!\;,\ \n\t]                	{ printf("%cannot%c",yytext[0],yytext[strlen(yytext)-1]); }
   [wW]on't[\.\?\!\;,\ \n\t]                	{ printf("%cill not%c",yytext[0],yytext[strlen(yytext)-1]); }
   I\ ain't[\.\?\!\;,\ \n\t]          	{ printf("I am not%c",yytext[strlen(yytext)-1]); }
   ([hH]e|[sS]he|[iI]t)\ ain't[\.\?\!\;,\ \n\t]  { printf("%s is not%c",strtok(yytext," "),yytext[strlen(yytext)-1]);}
   ([yY]ou|[tT]hey|[wW]e)\ ain't[\.\?\!\;,\ \n\t] { printf("%s are not%c",strtok(yytext," "),yytext[strlen(yytext)-1]);}
   [tT]hat\ ain't[\.\?\!\;,\ \n\t]       	{ printf("%chat is not%c",yytext[0],yytext[strlen(yytext)-1]); }
   [tT]his\ ain't[\.\?\!\;,\ \n\t]       	{ printf("%chis is not%c",yytext[0],yytext[strlen(yytext)-1]); }
   \'[cC]ause[\.\?\!\;,\ \n\t]              	{ printf("%cecause%c", yytext[1]-1,yytext[strlen(yytext)-1]); }
   'm[\.\?\!\;,\ \n\t]                      	{ printf(" am%c",yytext[strlen(yytext)-1]); }
   're[\.\?\!\;,\ \n\t]                     	{ printf(" are%c",yytext[strlen(yytext)-1]); }
   n't[\.\?\!\;,\ \n\t]                     	{ printf(" not%c",yytext[strlen(yytext)-1]); }
   'll[\.\?\!\;,\ \n\t]                     	{ printf(" will%c",yytext[strlen(yytext)-1]); }
   've[\.\?\!\;,\ \n\t]                     	{ printf(" have%c",yytext[strlen(yytext)-1]); }
   'd[\.\?\!\;,\ \n\t]                      	{ printf(" would%c",yytext[strlen(yytext)-1]); }
   's[\.\?\!\;,\ \n\t]                      	{ printf(" [is/has]%c",yytext[strlen(yytext)-1]); }
   %%
   int main(){
   	yylex();
   	int i=0;printf("\n\n");
   	FILE *fp;
   	fp= fopen("verbos.html", "w");
   	qsort (verbos, countVerbos, sizeof (const char *), compare);
   	fprintf(fp,"<html><body><h1>verbos:%d</h1><ul>\n",countVerbos);
   	for(i=0;i<countVerbos;i++)
       	fprintf(fp,"<li>to %s</li>\n",verbos[i]);
   	fprintf(fp,"</ul></body></html>");
   	fclose(fp);
   	return 1;
   }
\end{verbatim}

\bibliographystyle{alpha}
\bibliography{relprojLayout}





\end{document}
