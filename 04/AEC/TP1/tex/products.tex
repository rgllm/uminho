\section{Produtos no Mercado}\label{sec:Products}




\subsection{Redes Neuronais Artificiais}
No mercado começam a aparecer soluções na área de:
\begin{itemize}
\item Reconhecimento facial e tratamento de imagens 
(\url{ http://ieeexplore.ieee.org/abstract/document/554195/?reload=true }) 

\item Pesquisas ao nivel do DNA 
(\url{ http://www.pnas.org/content/88/24/11261.short} )

\item Analises metereologicas (\url{http://www.tandfonline.com/doi/abs/10.1080/02626669809492102}) 

\end{itemize}












\subsection{Aprendizagem por Reforço}
\begin{itemize}
\item DeepMind
    \begin{itemize}
    
    \item Rebenta os jogos da Atari...
    \url{https://deepmind.com/blog/deep-reinforcement-learning/}
    
    \end{itemize}

\item Mobileye
    \begin{itemize}
    
    \item The highway merging software was demoed in Barcelona by Mobileye, an Israeli automotive company that makes vehicle safety systems used by dozens of carmakers, including Tesla Motors.
    \url{https://www.technologyreview.com/s/603501/10-breakthrough-technologies-2017-reinforcement-learning/}
    
    \end{itemize}

\item OpenAI
    \begin{itemize}
    \item \url{https://openai.com/research/}
    \end{itemize}

\item Google
    \begin{itemize}
    \item Self driving car;
    \url{https://www.technologyreview.com/s/603501/10-breakthrough-technologies-2017-reinforcement-learning/}
    \end{itemize}

\item Uber
    \begin{itemize}
    \item Self driving car;
    \url{https://www.technologyreview.com/s/603501/10-breakthrough-technologies-2017-reinforcement-learning/}
    \end{itemize}

\end{itemize}










\subsection{Algoritmos genéticos}



Os Algoritmos Genéticos podem ser aplicados nas mais diversas áreas de desenvolvimento, como por exemplo:
\begin{itemize}
\item Área da \textbf{Música} - foi apresentado em 1999 na CEC99 (IEEE) um ambiente interativo, utilizando Algoritmos Genéticos para a avaliação de músicas (sequências de acordes) (10).
\item \textbf{Telecomunicações} - a US West, uma empresa de telecomunicações americana usou um sistema baseado em Algoritmos Genéticos que permitiam projetar, em cerca de duas horas, redes óticas especializadas, trabalho que levaria mais de seis meses caso fosse realizado por humanos (10).
\item \textbf{Medicina} - foram utilizados este tipo de algoritmos em 1999 para auxiliar na elaboração das escalas dos médicos de uma maternidade. O objetivo pretendido era ao mesmo tempo que se diminuía o esforço e desgate dos médicos garantir que havia disponibilidade 24h (10).
\item \textbf{Petróleo e Gás} - os Algoritmos Genéticos foram utilizados para o problema da inversão sísmica, bastante importante no campo da geologia e que consiste na determinação da estrutura dos dados de subsolo a partir da projeção geológica (10).
\end{itemize}