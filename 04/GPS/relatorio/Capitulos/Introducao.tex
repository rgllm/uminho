\chapter{Introdução}

O presente trabalho, realizado no âmbito da unidade curricular de \textit{Gestão de Processo de Software} tem como objetivo o desenvolvimento de uma metodologia de diagnóstico de um projeto Ágil. Inicialmente, de acordo com o planeado, o enunciado pretendia identificar se um determinado projeto exibia características que indicassem que estava a ser seguida uma metodologia Ágil contudo, e tendo em conta a aplicabilidade do projeto desenvolvido, foi decidido modificar o âmbito do projeto e tentar desenvolver uma metodologia que permitisse, à partida e antes de iniciar um projeto, saber se aquele projeto deveria ou não seguir uma metodologia Ágil.\\
Os métodos Ágeis, ou metodologias Ágeis são geralmente vistos como uma forma de evitar os métodos mais tradicionais, que nem sempre são a melhor opção para um projeto. Poucas são as organizações tecnicamente e psicologicamente capazes de adotar a 100\% uma abordagem Ágil de forma rápida e eficaz.\\
Numa primeira parte do projeto é descrito o contexto geral das abordagens Ágil, assim como a descrição concreta de vários métodos, entre eles Scrum, XP e Kanban. Numa segunda parte, e tendo em conta todo o conhecimento adquirido com a elaboração da primeira parte tenta-se agrupar e comparar os métodos Ágeis e tradicionais tendo em conta as suas características específicas, tentando levar o leitor a conseguir ter uma perceção dos pontos fortes e fracos das diversas abordagens. Por fim, e com todo este conhecimento é elaborado um questionário que ajuda a ter uma primeira perceção sobre que abordagem será mais adequada a seguir num determinado projeto que esteja prestes a começar, bem como a aplicação deste questionário a dois projetos: um académico e um de contexto real.