\section{Conclusões} \label{sec:Conclusao}

Tendo em conta o projeto desenvolvido e tudo o que foi apreendido durante o processo podemos tirar as seguintes conclusões:

\begin{itemize}
    \item Os processos de refactoring podem ser muitas vezes complexos quando o projeto já está numa fase muito adiantada;
    \item É importante ir corrigindo os diversos code smells à medida que o projeto está a ser desenvolvido;
    \item A revisão do código a cada iteração do projeto é de elevada importância;
    \item Nem todos os code smells têm um refactoring óbvio e nem sempre é possível corrigir o code smell e obter os mesmos resultados;
    \item Em geral, o tempo de execução melhorou com o refactoring dos code smells, tendo em conta também a sua natureza;
    \item É necessário ter em atenção à possível implementação de anti-padrões durante o  desenvolvimento de um projeto, tendo também em consideração que estes podem ter várias naturezas e estarem relacionados com várias etapas distintas do desenvolvimento;
    \item Tanto os vários anti-padrões como os code smells são, por vezes, bastante difíceis de detetar, só sendo possível melhorar a sua deteção e correção com a experiência.
\end{itemize}

Por tudo isto, o desenvolvimento deste projeto revela uma importância bastante elevada uma vez que foi assim possível aprimorar os nossos conhecimentos na área. 