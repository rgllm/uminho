\section{Ferramentas de desenvolvimento}\label{sec:Software}
\subsection{Redes Neuronais Artificiais}

Já existem várias frameworks para trabalhar com este tipo de redes no mercado, entre as quais se destaca a ferramenta \textbf{JustNN}(\url{http://www.justnn.com/}),que é uma ferramenta gratuita que funciona com vários tipos de ficheiros, tais como txt, csv, xls, bmp or ficheiros binários. É uma ferramenta bastante acessível para o utilizador e portanto usada em larga escala.
Além desta ferramenta, existem outras como é o caso da \textbf{OpenNN} (\url{http://www.opennn.net/}), uma ferramenta com um aspeto mais robusto que a anterior, opensource, desenvolvida em C++ e com ambiente gráfico ou a NeuralDesigner(https://www.neuraldesigner.com/) que permite ao utilizador uma utilização mais prática da biblioteca.










\subsection{Aprendizagem por Reforço}

\textbf{Piqle: a Generic Java Platform for Reinforcement Learning}

\begin{itemize}
\item \url{ https://sourceforge.net/projects/piqle/}
\end{itemize}

\textbf{"The Reinforcement Learning Toolbox"}

\begin{itemize}

\item \url{https://web.archive.org/web/20120722205525/http://www.igi.tugraz.at/ril-toolbox/general/overview.html}

\end{itemize}

\textbf{Java}

\begin{itemize}

\item \url{http://www.cse.unsw.edu.au/~cs9417ml/RL1/sourcecode.html}
\item \url{https://github.com/deeplearning4j/rl4j}
\item \url{http://burlap.cs.brown.edu/}

\end{itemize}


\textbf{Python}

\begin{itemize}
\item \url{https://github.com/openai/gym}
\end{itemize}

\textbf{Reinforcement Learning Glue}

\begin{itemize}
\item \url{http://glue.rl-community.org/wiki/Main_Page}
\end{itemize}










\subsection{Algoritmos genéticos}

Algumas das ferramentas disponíveis no mercado que usam AGs são:
\begin{itemize}
\item EvolveDotNet (Framework open-source para Algoritmos Genéticos - C\#)

\item GAlib (Framework open-source para Algoritmos Genéticos - C

\item GAUL (Biblioteca open-source para Algoritmos Genéticos e meta-heurísticas - C)

\item GeneticSharp(Biblioteca open-source e multiplataforma para Algoritmos Genéticos - C\#)

\item JAGA (Pacote open-source para Algoritmos Genéticos e Programação Genética - Java)

\item OpenBeagle (Biblioteca de Algoritmos Evolutivos e Genéticos - C

\item JGAP (Pacote open-source para Algoritmos Genéticos - Java)

\item jMetal(Framework open-source para otimização multi-objetivo que contém Algoritmos 
Genéticos - Java)

\item Pyevolve (Framework open-source para Algoritmos Genéticos e Programação Genética - Python)

\item Paradiseo(Framework para meta-heurísticas e algoritmos genéticos em C

\item KEEL (Ferramenta de software para extração de conhecimento)

\item ECJ (Sistema de pesquisa de computação evolutiva em Java)

\item Evolutionary Optimizer (Sistema de visualização de Algoritmos Genéticos)

\item GEATbx(Toolbox para Matlab com Algoritmos Evolutivos e Genéticos)

\item GPTIPS(Programação Genética e Data Mining para Matlab)

\item Pgapack(Biblioteca para Fortran e C para o uso de Algoritmos Genéticos)

\item PIKAIA (Software para o uso de Algoritmos Genéticos)

\item EVA2 (Workbench de Algoritmos Evolutivos)

\item AIGenetic(Biblioteca de Algoritmos Genéticos - Perl)
\end{itemize}