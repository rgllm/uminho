\chapter{Project Triggers}

\section{Proposta do Produto}

\subsection{Caso de Estudo}

O desporto rei ocupa uma posição fulcral na sociedade atua, sendo, muito provavelmente, uma das maiores fontes de entretenimento em todo o mundo. Podemos observar que muita da atenção dos media revolve em torno do futebol. A receita gerada é na ordem dos milhões e no seu mercado ocorrem, banalmente, transações milionárias. Face a isto, é fácil compreender que existam várias empresas, nomeadamente do ramo da informática que procuram um lugar no mercado, fazendo deste o seu principal negócio. Por exemplo, fornecendo aplicações e plataformas que permitem uma melhor gestão das equipas e dos seus bens. Um exemplo disto, é a plataforma desenvolvida pela \emph{F3M Information Systems S.A}, o \emph{TalentSpy}, onde é possível realizar \emph{scouting} pormenorizado de jogadores, através da criação e partilha de relatórios de observação, tendo como pedra basilar, uma base de dados muito completa, que é atualizada diariamente. 
De certa maneira, neste mercado, os jogadores são encarados como produtos, e por isso, existe a necessidade de vê-los como um todo, como atletas e como pessoas. Atendendo a este raciocínio, surge então uma oportunidade de negócio, a possibilidade de acompanhamento destes jogadores a diferentes níveis, por parte dos clubes ou de outras pessoas de interesse, tendo como suporte, a plataforma do \emph{TalentSpy} e a sua base de dados já existente.

\subsection{Objetivos do Produto}

O projeto tem como objetivo principal fazer um acompanhamento pessoal dos jogadores. Desta forma, pretende-se:
\begin{enumerate}
\item Aumentar a sua \emph{performance} dentro de campo, tendo em conta estatísticas de jogo;
\item Interagir com o jogador a um nível pessoal, podendo este interesse ser aliciante para novas propostas, ou apenas para o manter satisfeito;
\item Salvaguardar os interesses financeiros das entidades envolvidas nas transferências do jogador;
\item Salvaguardar os interesses financeiros do clube, em relação ao jogador.
\end{enumerate}

Para concretizar estes objetivos, desenhou-se um sistema de gestão de ocorrências que se pronta a notificar os utilizadores e que integra a plataforma \emph{TalentSpy}. Estas ocorrências podem ser da vida pessoal do jogador, da sua carreira, ou do seu clube.

\section{Clientes, consumidores e \emph{stakeholders}}

No desenvolvimento de um produto existem diversas partes envolvidas. Nem todas contribuem da mesma forma para a evolução e sucesso do mesmo. Podemos então dividi-las em 3 categorias, clientes, consumidores e ainda os \emph{stakeholders}.

\subsection{Clientes e Consumidores}
Em relação ao nosso produto o cliente e o consumidor são o mesmo, nomeadamente a \emph{F3M Information Systems S.A}, responsável pelo desenvolvimento da plataforma \emph{TalentSpy} e pela disponibilização da base de dados. A empresa tem uma função muito importante no desenvolvimento do produto, realizando diversas exigências e no final a aprovação do produto.
\subsection{\emph{Stakeholders}}
Entre as partes interessadas do nosso produto teremos os \textbf{jogadores}, que são o alvo principal do sistema. Estes desempenham um papel fulcral para o desenvolvimento da aplicação, uma vez que, mesmo não tendo um papel direto na sua utilização, a aplicação não existiria sem os seus dados.
Os \textbf{agentes}, são responsáveis pela gestão dos jogadores a nível comercial e pessoal, procurando satisfazer as suas necessidades particulares e desportivas, através da obtenção de melhores contratos.
Os \textbf{treinadores}, são responsáveis pela gestão desportiva dos jogadores, e procuram obter o melhor rendimento de cada um, em campo.
Os dirigentes desportivos dos clubes têm os interesses financeiros nos jogadores, preocupando-se com as transferências, prémios, direitos e cláusulas dos jogadores.
E por fim, os \textbf{jornalistas} têm interesse nas estatísticas dos jogadores, seja para produzirem artigos mais completos e fidedignos ou para, por exemplo, apresentarem dados estatísticos relevantes durante o relato de um jogo de futebol.

\section{Utilizadores do Produto}

Nesta secção vão ser categorizados e agrupados os diferentes futuros utilizadores do produto. Irão ser atribuídos nomes a cada um dos grupos e especificadas responsabilidades, papéis e experiência, bem como classificações de cada um destes perante determinados parâmetros. Nesta etapa de desenvolvimento do produto, foi identificado 1 tipo de utilizadores: utilizadores padrão.

Os utilizadores padrão do sistema de ocorrências são principalmente, os agentes dos jogadores, os treinadores, os gestores desportivos dos clubes e os jornalistas. Estas entidades estão intrinsecamente ligadas ao ambiente futebolístico, como abordado nos tópicos anteriores.

Estes utilizadores vão desempenhar no sistema, tarefas como consultar as estatísticas dos jogadores, de jogo: como cartões, número de golos, assistências, jogos; de carácter pessoal: como datas de aniversário de familiares, compromissos. Os utilizadores vão poder também criar novas ocorrências, preenchendo os dados que possam estar em falta, de ocorrências já existentes. E por fim, os utilizadores vão poder criar novos tipos de ocorrências, que não existam no sistema.

Os utilizadores padrão são expectáveis de terem uma experiência tecnológica suficiente para operarem o sistema, uma vez que é muito provável que já utilizem o \emph{TalentSpy}, plataforma que apontamos integrar, e desenvolverem um conhecimento moderado acerca do sistema, com apenas algumas utilizações.
