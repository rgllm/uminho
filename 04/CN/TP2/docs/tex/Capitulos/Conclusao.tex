\chapter{Trabalho Futuro}

Em contextos reais de exploração de datasets, através de algoritmos de \textit{Machine Learning}, muitos dos conjuntos de dados encontram-se em bruto e com bastante ruído pelo meio. Neste projeto o foco principal não se encontra na extração de conhecimento deste dataset em especifico, mas sim na criação de uma RNA capaz de realizar a classificação de publicações. 

Nesse sentido, poderia ser relevante reunir um conjunto de publicações da rede social \textit{twitter} através de diversas APIs disponibilizadas na linguagem \textit{Python}, sendo assim possível filtrar desde inicio o contexto da publicação e o idioma em que a mesma é recolhida. 

Este aspeto é frisado por se ter observado apesar do dataset ter um grande volume de dados, a maioria dos exemplos são vagos e associados apenas a eventos ou referências a \textit{websites}, não apresentando assim qualquer tipo de informação sentimental. 
Das 990472 instâncias utilizadas do dataset, cerca de 92\% foram classificadas como neutras pela API utilizada para criar os atributos \textit{target}, sobre os quais a rede desenvolve o seu processo de aprendizagem supervisionado. 

Por este motivo, uma melhor seleção dos \textit{Tweets} recolhidos ou uma redução do dataset para equilibrar o número de casos neutros com os casos associados a sentimentos negativos/positivos será uma possível mais valia para o projeto e capacidade de previsão das RNAs criadas. 

Ainda assim, no sentido de tirar o melhor partido do dataset fornecido, foi realizada uma redução das dimensões do mesmo. Desta modo foi possível focar em especifico apenas publicações que se encontrassem em inglês. 
Uma vez que a API de análise sentimental de texto e representação de frases em vetores de palavras está essencialmente focada na língua inglesa, procurou-se assim melhorar a qualidade dos dados fornecidos à rede. 

Ao utilizar um dataset de menores dimensões mas ainda assim significativo (100 mil instâncias), foi possível reduzir o tempo de aprendizagem e recursos necessários para executar o treino da RNA. Com estas vantagens, foi possível explorar mais configurações e topologias de RNAs, no sentido de obter melhores resultados. 

Uma vez que uma RNA não está desenvolvida para ser "alimentada" com conteúdo textual na forma de \textit{Strings}, outro ponto chave deste projeto está ligado com a representação de texto num formato capaz de ser interpretado e fornecido de forma coerente à camada de \textit{input} da rede. 

Neste projeto a representação de uma frase recai num vetor de palavras, sendo que cada índice do vetor representa uma determinada palavra. Esta representação está limitada a nível das dimensões do vetor, principalmente se a cada \textit{Epoch} da rede for necessário ter preparados um número elevado (>500) de instâncias para processar numa iteração de treino. 

Nos testes realizados foram utilizados vetores com 3500 posições, referentes às 3500 palavras mais utilizadas na língua inglesa. Contudo, outras formas de representação textual poderiam ser exploradas, como \textit{N-grams}, no sentido de comparar os resultados obtidos a nivel de capacidade de previsão da RNA. 

Contudo, devido ao espetro temporal do projeto, a equipa teve que se manter apenas com a representação inicialmente construida para representar texto.

Outra técnica bastante utilizada em contextos de \textit{Machine Learning} passa por utilizar arquiteturas previamente criadas e com bons resultados estudados num contexto semelhante. 
Nesse sentido, a abordagem realizada por Pedro M. Sosa em \cite{PedroCNN} poderia ser um bom ponto de partida, se o seu trabalho fosse disponibilizado com mais detalhe. 
Na sua abordagem, o uso de redes convulocionais, com ligações recorrentes, permite ter em conta o contexto da frase num todo, dado que palavras já "processadas" influenciam as iterações posteriores, com as palavras seguintes do mesmo input. 


\chapter{Conclusão}
\label{chp:conclusao}

No sentido de desenvolver uma Rede Neuronal Artificial capaz de realizar a classificação de texto em níveis emocionais distintos, os capítulos anteriores apresentam todas as etapas desenvolvidas ao longo deste projeto. 

O capitulo inicial, onde se introduzem os principais conceitos associados com RNAs, apresenta-se essencial, no sentido em que foca todos os aspetos necessários para compreender muitas das decisões tomadas ao longo do projeto, relacionadas com o modo de funcionamento e aprendizagem destas estruturas. 

Relativamente ao dataset utilizado, destaca-se a dificuldade encontrada na fase de pré processamento, até ser possível obter informação útil e totalmente "limpa" do mesmo. 
Só depois de converter os textos das publicações num formato uniforme, quer a nível de codificação de caracteres como de linguagem, é que foi possível aplicar métodos para determinar o valor sentimental de cada publicação e ser-se capaz de adaptar os \textit{tweets} para um formato de \textit{input} capaz de alimentar as RNAs construidas.

A nível dos padrões de treino executados, foram explorados praticamente todos os aspetos possíveis de configurar numa RNA. Estas configurações foram realizadas de acordo com o contexto do problema e objetivos a alcançar, focando as principais técnicas utilizadas atualmente. 

Neste processo destaca-se a exploração de diferentes topologias, métodos de inicialização de pesos e funções de ativação, como as atuais variações da função ReLU. 
No geral, quer utilizando o dataset completo quer utilizando uma versão reduzida e exclusivamente com \textit{Tweets} em inglês, sem traduções posteriores, verifica-se que a accuracy média a nível da fase de teste atinge os 88\%. Qualquer um dos modelos propostos, baseados neste dataset com as limitações referidas ao longo do relatório, fica assim "fixo" a esta capacidade de previsão. 

A conjugação dos diferentes parâmetros anteriormente referidos, permite atenuar ligeiramente efeitos de overfitting aos dados de treino. 
Contudo, a capacidade de previsão da rede mantém-se bastante semelhante em qualquer um dos diversos cenários de Treino/Teste realizados e interpretados ao longo do Capitulo \ref{cht:analiseresultados}. 

Em suma, apesar da qualidade de um processo de treino de uma RNA estar intrinsecamente ligado com os parâmetros e técnicas utilizadas na fase de aprendizagem, destaca-se a importância que a qualidade dos dados utilizados também desempenham na fase de treino. 
Como consequência deste aspeto, apesar do tratamento do dataset ter sido uma parte constante ao longo do projeto, outras formas de representação de texto e cálculo do valor sentimental de cada \textit{Tweet} seriam relevantes de explorar para obter resultados mais sólidos de classificação por parte das RNAs criadas. 

