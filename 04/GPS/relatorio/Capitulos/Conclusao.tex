\chapter{Conclusão e Trabalho Futuro}
\label{chp:conclusao}

Depois de analisados e descritos os princípios e algumas metodologias Ágeis, constata-se que estas trouxeram melhorias notórias nos processos e fluxos de trabalho das empresas, que se convertem em maior organização, capacidade de produção e consequentemente, satisfação dos clientes. É de notar também que, apesar do que muitas pessoas possam imaginar, estas metodologias não são propriamente recentes. Já foram inventadas há alguns anos, mas só começaram a popularizar-se na área da produção de Software há relativamente pouco tempo e o seu uso tem vindo a tornar-se cada vez mais comum. Apesar das vantagens das abordagens mais Ágeis, elas têm vantagens e desvantagens e não servem para todo o tipo de projeto.\\
Na nossa opinião o método desenvolvido tem algumas características que à partida definimos como prioritárias: é simples, não demora muito a responder, é direto e ajuda a ter uma primeira perceção do método de trabalho a seguir.\\
É também importante definir que esta é apenas uma primeira versão do método de diagnóstico, será esperado que como trabalho futuro se sugira uma segunda parte do questionário que apenas seja respondida se a primeira parte indicar que o projeto deva seguir uma abordagem ágil, e onde se consiga perceber em específico que método Ágil o projeto deve seguir. Para além disso, e devido ao facto de o desenvolvimento deste projeto ter sido feito num contexto académico com prazos limitados, é também importante validar as questões num contexto real em projetos já desenvolvidos e onde se consiga ter uma perceção se as abordagens ágeis ou tradicionais resultaram.
