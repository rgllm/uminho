\chapter{\emph{Persona}}

Uma das técnicas de levantamento de requisitos é o uso de \emph{personas}, onde são criadas personagens fictícias que representam um tipo de utilizador importante do produto em desenvolvimento. Uma \emph{persona} é um protótipo das pessoas que fazem parte do público-alvo, devendo ser concebida para representar, naquilo que é essencial e distintivo, essas pessoas.
Os utilizadores para quem o produto se destina são os agentes pessoais dos jogadores, treinadores de futebol, administradores e gestores de um clube de futebol.


\section{Ricardo Matos, o treinador perfecionista}

\begin{figure}[H]
    \includegraphics[scale=1]{img/treinador.png}
\end{figure}

\noindent\textbf{Idade:} 43 anos
\newline\textbf{Estado Civil:} Casado
\newline\textbf{Habilitações Académicas:} Mestre em Treino de Alto Rendimento Desportivo pela Faculdade de Desporto da Universidade do Porto.
\newline\textbf{Profissão:} Treinador principal do Sport Clube de Mirandela
\newline\textbf{Residência:} Mora numa moradia perto do Estádio São Sebastião.
\vspace{3mm}
\newline\textbf{Estilo de vida:} Ricardo, um natural de Mirandela, antigo jogador do Sport Clube de Mirandela e fã de desporto, obteve o grau de mestre no curso de com o intuito de treinar o clube do coração. É um homem de família, com dois filhos com quem adora fazer todo o tipo de desporto. Isso não o impede de ter uma relação muito próxima com os jogadores, organizando regularmente festas na sua casa, integrando assim totalmente a sua vida pessoal com a profissional. Em termos do seu cargo como treinador, o Ricardo é um perfecionista, apesar de ser amigo dos jogadores exige pontualidade e disciplina, fatores que levaram ao sucesso do seu trabalho refletindo-se na subida de divisão do clube.
\vspace{3mm}
\newline\textbf{Contexto de utilização do produto:} O Ricardo sendo treinador de futebol é responsável por avaliar em cada treino que conduz se um jogador está apto para aquilo que pretende para a equipa. Apesar de se esforçar por se envolver na vida pessoal dos jogadores, nem sempre consegue dar um acompanhamento igual a todos. Para isso vai perguntando constantemente a colegas sobre como vai a vida pessoal de um jogador, sempre que este se apresenta com um rendimento abaixo do esperado. Gosta de motivar os jogadores realçando estatísticas positivas sobre as suas carreiras.
\vspace{3mm}

\textbf{Objetivos:}
\begin{enumerate}
    \item Ser informado sobre diversos aspetos da vida pessoal de um jogador, como por exemplo:
    \begin{enumerate}
        \item Aniversário do jogador;
        \item Aniversário dos filhos;
        \item Morte do pai, mãe, entre outros relevantes.
    \end{enumerate}
    \item Ser informado sobre aspetos da carreira profissional de um jogador, como por exemplo:
    \begin{enumerate}
        \item Número de golos marcados em todas as competições;
        \item Cartões amarelos;
        \item Número de faltas cometidas;
    \end{enumerate}
    \item Introduzir novos tipos de ocorrências que ache relevante ser relembrado no futuro.
    \item Remover ocorrências que já não ache relevante ser avisado.
\end{enumerate}

\section{José Castro, o agente amigo}

\begin{figure}[H]
    \includegraphics[scale=1]{img/agente.png}
\end{figure}

\noindent\textbf{Idade:} 33 anos
\newline\textbf{Estado Civil:} Solteiro
\newline\textbf{Habilitações Académicas:} Mestre em Gestão pela Faculdade de Economia da Universidade do Porto.
\newline\textbf{Profissão:} Agente desportivo
\newline\textbf{Residência:} Mora num T2 na foz do Porto
\vspace{3mm}
\newline\textbf{Estilo de vida:} José gosta muito de viajar e de fazer desporto. Como ex-jogador de futebol amador a sua preferência recai para o futebol. Apesar de ter diversos jogadores espalhados pelo país como clientes a sua residência fixa é no Porto, o que obriga a constantes deslocações para se reunir com os seus clientes. Profissionalmente, José é uma agente que faz o esforço extra para conseguir as melhores condições possíveis para os seus representados. Também se preocupa com a vida pessoal dos mesmos, realizando chamadas telefónicas quase diárias.
\vspace{3mm}
\newline\textbf{Contexto de utilização do produto:} Devido à natureza do seu trabalho é responsável por se inteirar das intenções dos seus clientes e escrutinar qualquer hipotético problema que um dos seus jogadores tenha e que lhe esteja a afetar a performance. 
Como negociador dos contratos tem também de ter argumentos para valorizar o trabalho dos seus clientes, isto com recurso a estatísticas da carreira.
\vspace{3mm}

\textbf{Objetivos:}
\begin{enumerate}
    \item Ser informado sobre diversos aspetos da vida pessoal de um cliente, como por exemplo:
    \begin{enumerate}
        \item Aniversário do jogador;
        \item Aniversário dos filhos;
        \item Morte do pai, mãe, entre outros relevantes.
    \end{enumerate}
    \item Ser informado sobre aspetos da carreira profissional de um jogador, como por exemplo:
    \begin{enumerate}
        \item Número de golos marcados em todas as competições;
        \item Expiração de contrato;
        \item Valor do mercado;
        \item Número de jogos pelo clube;
    \end{enumerate}
    \item Ser informado sobre aspetos presentes nos contratos dos seus clientes, como por exemplo:
    \begin{enumerate}
        \item Alerta de fim de empréstimo;
        \item Alerta de clausula de compra no final do empréstimo;
        \item Alerta de prémios contidos no contrato;
        \item Alerta de pagamento por jogos;
    \end{enumerate}
    \item Introduzir ocorrências que ache relevante ser relembrado no futuro.
    \item Remover ocorrências que já não ache relevante ser avisado.
\end{enumerate}

\newpage

\section{Alberto Martins, o gestor implacável}

\begin{figure}[H]
    \includegraphics[scale=1]{img/Picture1.png}
\end{figure}

\noindent\textbf{Idade:} 40 anos
\newline\textbf{Estado Civil:} Casado
\newline\textbf{Habilitações Académicas:} Mestre em Gestão pela Universidade do Minho.
\newline\textbf{Profissão:} Gestor do Valadares Gaia Futebol Clube.
\newline\textbf{Residência:} Mora num T3 na periferia de Vila Nova de Gaia.
\vspace{3mm}
\newline\textbf{Estilo de vida:} Alberto é um homem conservador e que gosta de passar bastante tempo em casa a contemplar o seu jardim. Gosta muito da sua família, mas não tem muitos amigos nem se esforça para os ter. Portanto a sua relação com os funcionários do seu clube é estritamente profissional. Nesse aspeto, Alberto tem sucesso, sendo que a saúde financeira do seu clube é bastante boa.
\vspace{3mm}
\newline\textbf{Contexto de utilização do produto:} Sendo o gestor do clube, tem mais trabalho que o próprio presidente. É ele o responsável por negociar transferências de jogadores, renovar contratos e efetuar todo o tipo de pagamentos. Tem por isso a necessidade de estar bem informado sobre todos os aspetos do seu clube. 

\vspace{3mm}

\textbf{Objetivos:}
\begin{enumerate}
    \item 1.	Ser informado sobre diversos aspetos da vida pessoal de um jogador, como por exemplo:
    \begin{enumerate}
        \item Aniversário do jogador;
    \end{enumerate}
    \item Ser informado sobre aspetos da carreira profissional de um jogador, como por exemplo:
    \begin{enumerate}
        \item Expiração de contrato;
        \item Valor de mercado;
        \item Número de jogos pelo clube;
    \end{enumerate}
    \item Ser informado sobre aspetos presentes nos contratos dos seus jogadores, como por exemplo:
    \begin{enumerate}
        \item Alerta de fim de empréstimo;
        \item Alerta de clausula de compra no final do empréstimo;
        \item Alerta de prémios contidos no contrato;
        \item Alerta de pagamento por jogos;
    \end{enumerate}
    \item Ser informado sobre aspetos relativos a direitos de transferência de ex-jogadores do seu clube que possam gerar retorno financeiro, bem como direitos de formação e solidariedade;
    \item Introduzir ocorrências que ache relevante ser relembrado no futuro. 
    \item Remover ocorrências que já não ache relevante ser avisado.
\end{enumerate}