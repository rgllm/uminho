%
% Layout retirado de http://www.di.uminho.pt/~prh/curplc09.html#notas
%
\documentclass{report}
%encoding
%--------------------------------------
\usepackage[utf8]{inputenc}
\usepackage[T1]{fontenc}
%--------------------------------------

%Portuguese-specific commands
%--------------------------------------
\usepackage[portuguese]{babel}
%--------------------------------------

%Hyphenation rules
%--------------------------------------
\usepackage{hyphenat}
\hyphenation{mate-mática recu-perar}
%--------------------------------------

\usepackage{url}
\usepackage{enumerate}
\usepackage{graphicx}

%\usepackage{alltt}
%\usepackage{fancyvrb}
\usepackage{listings}
%LISTING - GENERAL
\lstset{
    basicstyle=\small,
    numbers=left,
    numberstyle=\tiny,
    numbersep=5pt,
    breaklines=true,
    frame=tB,
    mathescape=true,
    escapeinside={(*@}{@*)}
}
%
%\lstset{ %
%   language=C,                          % choose the language of the code
%   basicstyle=\ttfamily\footnotesize,      % the size of the fonts that are used for the code
%   keywordstyle=\bfseries,                 % set the keyword style
%   %numbers=left,                          % where to put the line-numbers
%   numberstyle=\scriptsize,                % the size of the fonts that are used for the line-numbers
%   stepnumber=2,                           % the step between two line-numbers. If it's 1 each line
%                                           % will be numbered
%   numbersep=5pt,                          % how far the line-numbers are from the code
%   backgroundcolor=\color{white},          % choose the background color. You must add \usepackage{color}
%   showspaces=false,                       % show spaces adding particular underscores
%   showstringspaces=false,                 % underline spaces within strings
%   showtabs=false,                         % show tabs within strings adding particular underscores
%   frame=none,                             % adds a frame around the code
%   %abovecaptionskip=-.8em,
%   %belowcaptionskip=.7em,
%   tabsize=2,                              % sets default tabsize to 2 spaces
%   captionpos=b,                           % sets the caption-position to bottom
%   breaklines=true,                        % sets automatic line breaking
%   breakatwhitespace=false,                % sets if automatic breaks should only happen at whitespace
%   title=\lstname,                         % show the filename of files included with \lstinputlisting;
%                                           % also try caption instead of title
%   escapeinside={\%*}{*)},                 % if you want to add a comment within your code
%   morekeywords={*,...}                    % if you want to add more keywords to the set
%}

\usepackage{xspace}

\parindent=0pt
\parskip=2pt

\setlength{\oddsidemargin}{-1cm}
\setlength{\textwidth}{18cm}
\setlength{\headsep}{-1cm}
\setlength{\textheight}{23cm}

\def\darius{\textsf{Darius}\xspace}
\def\antlr{\texttt{AnTLR}\xspace}
\def\pl{\emph{Processamento de Linguagens}\xspace}

\def\titulo#1{\section{#1}}
\def\super#1{{\em Supervisor: #1}\\ }
\def\area#1{{\em \'{A}rea: #1}\\[0.2cm]}
\def\resumo{\underline{Resumo}:\\ }


%%%%\input{LPgeneralDefintions}

\title{Processamento de Linguagens (3º ano do MiEI)\\ \textbf{Trabalho Prático 2}\\ Compilador usando Flex e Yacc - Linguagem Tuga}
\author{Gustavo Andrez\\ (A27748) \and Rogério Moreira\\ (A74634) \and Samuel Ferreira\\ (A76507) }
\date{\today}

\begin{document}

\maketitle

\begin{abstract}

O presente trabalho tem como objetivo aumentar a experiência em engenharia de linguagens e em programação generativa (gramatical), com o propósito de desenvolver processadores de linguagens segundo o método da tradução dirigida pela sintaxe, a partir de uma gramática tradutora, para uma máquina de stack virtual.
A ferramenta usada foi o Yacc. Todos os objetivos inicialmente propostos foram cumpridos.

\end{abstract}

\tableofcontents

\chapter{Introdu\c{c}\~ao} \label{intro}

\section*{Introdu\c{c}\~ao} \

O presente relatório tem como objetivo documentar o processo de desenvolvimento de uma linguagem de programação imperativa simples, e o respetivo compilador, que deverá ser capaz de gerar pseudo-código Assembly para máquina virtual VM, utilizada neste projeto. Para este fim, é necessário criar uma gramática independente de contexto que defina a linguagem, e estabelecer as regras de tradução para o Assembly da VM fornecido pela equipa docente.
O grupo decidiu criar uma linguagem em que todas as instruções são descritas em português de Portugal, apelidada de \emph{Tuga}. 

\section*{Enunciado} \

Pretende-se que se comece por definir uma linguagem de programação imperativa simples.
Deve-se ter em consideração que essa linguagem terá de permitir:
\begin{itemize}
\item declarar e manusear variáveis atómicas do tipo inteiro, com os quais se podem realizar as habituais operações aritméticas, relacionais e lógicas;
\item declarar e manusear variáveis estruturadas do tipo array (a 1 ou 2 dimensões) de inteiros, em relação aos quais é apenas permitida a operação de indexação (índice inteiro);
\item efetuar instruções algorítmicas básicas como a atribuição de expressões a variáveis;
\item ler do standard input e escrever no standard output.
\item efetuar instruções para controlo do fluxo de execução — condicional e cíclica — que possam ser aninhadas;
\item as variáveis deverão ser declaradas no início do programa e não pode haver re-declarações, nem utilizações sem declaração prévia.  Se nada for explicitado, o valor da variável após a declaração é 0 (zero);
\end{itemize}
Pretende-se então desenvolver um compilador para a linguagem definida com base na GIC criada acima e com recurso ao Gerador Yacc/Flex.
O compilador deve gerar pseudo-código Assembly da Máquina Virtual VM.

\section*{Descrição do problema}
O problema proposto no segundo trabalho prático da disciplina de Processamento de Linguagens compreende a realização de uma linguagem LIPS, com uma gramática própria e capaz de ser executada numa máquina virtual com Assembly próprio previamente fornecida pela equipa docente.
Posto isto, o problema tem três fases principais:
\begin{enumerate}
\item Desenho de uma gramática tradutora (Yacc);
\item Tradução das instruções da gramática para o Assembly da máquina virtual (Flex);
\item Elaboração de programas de teste na linguagem elaborada;
\end{enumerate}


\section*{Estrutura do Relatório} \

O relatório está dividido em três capítulos, correspondentes ao desenho da linguagem e gramática, às regras de tradução criadas,
e ao resultado dos testes efetuados com programas exemplo, por esta ordem. 
No primeiro capítulo, \emph{Desenho da linguagem/gramática}, são expostos tanto os requisitos do problema  apresentado, e discutidas as estratégias utilizadas para a consequente implementação da solução do problema. O capítulo dois, \emph{Regras da tradução para Assembly da VM}, apresenta para além das operações de tradução da linguagem para o Assembly próprio da VM, são também descritas a especificação das estruturas de dados e outras notas importantes.
Por fim, o capítulo \emph{Testes e Resultados} expõe o resultado dos testes requeridos à demonstração do funcionamento da linguagem e compilador desenvolvidos durante o projeto. Para isto foram criados alguns programas testes como pedido no enunciado do trabalho prático.

\chapter{Desenho da linguagem/gramática} \label{ae}

\section{Símbolos terminais e não terminais}

\begin{verbatim}
T={INT , VAR , OPINC , OPDEC , 
OPATRMAIS , OPATRMENOS , 
OPATRPROD , OPATRDIV , OPATRMOD , OPAND , 
OPOR , OPGT , OPLT , OPGE , OPLE , OPEQ , OPNE ,
OPARITMAIS , OPARITMENOS , OPARITPROD ,
OPARITDIV , OPARITMOD , STRING , SCAN , 
PRINT , IF , ELSE , NOT , ERRO , WHILE , 
( , ) , [ , ] , { , } , = , " , ; , & }
 
 OPINC ::- \+\+    
 OPDEC ::- \-\-    
 
 OPATRMAIS ::- \+\=    
 OPATRMENOS ::- \-\=    
 OPATRPROD ::- \*\=     
 OPATRDIV ::- \/\=     
 OPATRMOD ::- \%\=    
 
 SCAN ::- (?i:ler)
 
 PRINT ::- (?i:escrever)
 STRING ::- \"([^"]|\\\")*\"
 
 IF ::- (?i:se)   
 ELSE ::- (?i:senao)
 WHILE ::- (?i:enquanto)
 
 OPGE ::- \>\=
 OPLE ::- \<\=
 OPEQ ::- \=\=
 OPNE ::- \!\=
 NOT ::- \!    
 OPGT ::- \>    
 OPLT ::- \<    
 
 
 OPAND ::- \&\&
 OPOR ::- \|\|
 
 OPARITMENOS ::- \-    
 OPARITMAIS ::- \+    
 OPARITPROD ::- \*    
 OPARITDIV ::- \/    
 OPARITMOD ::- \%    
 
 INT ::- [0-9]+
 VAR ::- [a-z_][a-zA-Z0-9_]*     
 
 ERRO ::- .
\end{verbatim}
 
 
 
O conjunto dos símbolos não terminais da gramática é o que se segue:
 
\begin{verbatim}
NT={ Programa , Declaracoes , Instrucoes , 
Declaracao , Instrucao , Atribuicao , 
Escrita , Leitura , Condicional , 
Ciclo , Exp }
\end{verbatim}

\section{Gramática}

Na linguagem desenhada os programas seguem a seguinte estrutura: nas primeiras linhas são declaradas todas as variáveis (inteiros ou arrays de inteiros) que irão ser necessárias durante o programa, depois aparece o símbolo ‘&’ e por fim as instruções do programa. (Programa : Declaracoes '\&' Instrucoes).

A gramática elaborada é a seguinte:
 
\begin{verbatim}
//--- CABECA ---//
 
Declaracoes : Declaracoes Declaracao
       |
         ;
 
Declaracao : VAR ';'
      | VAR '[' INT ']' ';'
      ;
 
//--- CORPO ---//
Instrucoes : Instrucoes Instrucao
      | Instrucao
      ;
 
Instrucao : Atribuicao
       | Escrita
       | Leitura
       | Condicional
       | Ciclo
       ;
 
//--- ATRIBUICAO ---//
Atribuicao : VAR '=' Exp ';'
      | VAR '[' Exp ']' '=' Exp ';
      | VAR OPINC ';'
      | VAR OPDEC ';'
      | VAR OPATRMAIS Exp ';'
      | VAR OPATRMENOS Exp ';'
      | VAR OPATRPROD Exp ';'
        | VAR OPATRDIV Exp ';'
        | VAR OPATRMOD Exp ';'
      ;
 
//--- LEITURA ---//
Leitura : SCAN VAR ';'
     | SCAN VAR '[' Exp ']' ';'
     ;
 
//--- ESCRITA ---//
Escrita : PRINT Exp ';'
     | PRINT STRING ';'
     ;
 
//--- CONDICIONAL IF ---//
Condicional : IF '(' Exp ')' '{' Instrucoes '}' Else
       ;
 
Else : ELSE '{' Instrucoes '}'
     |  
     ;
 
//--- CICLO WHILE ---//
Ciclo : WHILE '(' Exp ')' '{' Instrucoes '}'
      ;
 
//--- EXPRESSAO ---//
Exp : NOT Exp
    | Exp OPLT Ex
    | Exp OPGT Exp
    | Exp OPLE Exp
    | Exp OPGE Exp
    | Exp OPEQ Exp
    | Exp OPNE Exp
    | Exp OPAND Exp
    | Exp OPOR Exp
    | Exp OPARITMAIS Exp
    | Exp OPARITMENOS Exp
    | Exp OPARITPROD Exp
    | Exp OPARITDIV Exp
    | Exp OPARITMOD Exp
    | VAR
    | VAR '[' Exp ']'
    | INT
    | '(' Exp ')'
    ;
 
\end{verbatim}


\chapter{Regras da tradução para Assembly da VM} \label{ae}

Em seguida é apresentado um pseudocódigo que descreve o processo de conversão da linguagem desenvolvida (Tuga) para linguagem Assembly da VM.

\begin{verbatim}
Programa : Declaracoes { adicionar instrução "start" à pilha de instruções }
       '&' Instrucoes { adicionar instrução "stop" à pilha de instruções }
        ;
 
Declaracoes : Declaracoes Declaracao
       |
         ;
 
Declaracao : VAR ';' {
         se (o nome de variável em $1 já foi atribuído)
           erro
         senao
           registar variável na estrutura de dados (nome, tipo, endereço)
           adicionar instrução "pushi 0" à pilha de instruções
           gp++
         }
      | VAR '[' INT ']' ';' {
           se (o nome de variável em $1 já foi atribuído)
           erro
         senao
           registar variável na estrutura de dados (nome, tipo, endereço)
           adicionar instrução "pushn X" à pilha de instruções, em que X=$3
           gp+=$3
           }
      ;
 
Instrucoes : Instrucoes Instrucao
      | Instrucao
      ;
 
Instrucao : Atribuicao
       | Escrita
       | Leitura
       | Condicional
       | Ciclo
       ;
 
Atribuicao : VAR '=' Exp ';' {
           se(não existe a variável em $1 || a variável é do tipo inteiro)
             erro
           senao
             adicionar a instrução "storeg X" à pilha de instruções, em que X=endereço($1)
         }
      | VAR {
             se(não existe a variável em $1 || a variavel é do tipo array)
             erro
           senao
             adicionar a instrução "pushgp" à pilha de instruções
             adicionar a instrução "pushi X" à pilha de instruções, em que X=endereço($1)
             adicionar a instrução "padd" à pilha de instruções
         }
       '[' Exp ']' '=' Exp ';'{
             adicionar a instrução "storen" à pilha de instruções
           }
 
      | VAR OPINC ';'{
             se(não existe a variável em $1 || a variavel é do tipo array)
             erro
           senao
             // incrementar a variável $1
             adicionar a instrução "pushg X" à pilha de instruções, em que X=endereço($1)
             adicionar a instrução "puhi 1" à pilha de instruções
             adicionar a instrução "add" à pilha de instruções
             adicionar a instrução "storeg X" à pilha de instruções, em que X=endereço($1)
           }
      | VAR OPDEC ';'{ processo análogo ao anterior, mas decrementa em vez de incrementar    }
     (1)->  | VAR {
             se(não existe a variável em $1 || a variavel é do tipo array)
             erro
           senao
             adicionar a instrução "pushg X" à pilha de instruções, em que X=endereço($1)
         }
       OPATRMAIS Exp ';'{
           adicionar a instrução "add" à pilha de instruções
           adicionar a instrução "storeg X" à pilha de instruções, em que X=endereço($1)
           }
        | VAR OPATRMENOS Exp ';'{ análogo ao processo em (1) }
        | VAR OPATRPROD Exp ';'{ análogo ao processo em (1) }
        | VAR OPATRDIV Exp ';'{ análogo ao processo em (1) }
        | VAR OPATRMOD Exp ';'{ análogo ao processo em (1) }
      ;
 
Leitura : SCAN VAR ';' {
         se(não existe a variável em $1 || a variavel é do tipo array)
           erro
         senao
           adicionar a instrução "read" à pilha de instruções
           adicionar a instrução "atoi" à pilha de instruções
           adicionar a instrução "storeg X" à pilha de instruções, em que X=endereço($2)
       }
     | SCAN VAR {
         se(não existe a variável em $1 || a variável é do tipo inteiro)
             erro
         senao
           adicionar a instrução "pushgp" à pilha de instruções
           adicionar a instrução "pushi X" à pilha de instruções, em que X=endereço($2)
           adicionar a instrução "padd" à pilha de instruções
       }
     '[' Exp ']' ';'    {
         adicionar a instrução "read" à pilha de instruções
         adicionar a instrução "atoi" à pilha de instruções
         adicionar a instrução "storen" à pilha de instruções
       }
     ;
 
Escrita : PRINT Exp ';' { adicionar a instrução "writei" à pilha de instruções }
     | PRINT STRING ';'{       
         adicionar a instrução "pushs X" à pilha de instruções, em que X= endereço($2)
         adicionar a instrução "writes" à pilha de instruções
       }
     ;
 
Condicional : IF '(' Exp ')' {
           adicionar endereço da instrução à stack de se's
           adicionar a instrução "jz ifX" à pilha de instruções, em que X=ifs
         }
        '{' Instrucoes '}' { fazer pop da stack de se's }
         Else
       ;
 
Else : ELSE {  
         adicionar endereço da instrução à stack de se's
         adicionar a instrução "jump elseX" à pilha de instruções, em que X=elses
         adicionar a label "ifX:" à pilha de instruções, em que X= ifs
         incrementar ifs
       }
     '{' Instrucoes '}' {
         fazer pop da stack de se's
         adicionar a label "elseX:" à pilha de instruções, em que X=elses
         incrementar variável elses
       }
     |  {    adicionar a label "ifX:" à pilha de instruções, em que X=ifs
       incrementar variável ifs }
     ;
 
Ciclo : WHILE '(' {
         $1=whiles;
         incrementar a variável ""whiles" em 2 unidades
         adicionar a label "whileX :" à pilha de instruções, em que X=$1
       }
      Exp ')' {  
         adicionar a instrução "jz whileX" à pilha de instruções, em que X=$1+1
       }
      '{' Instrucoes {
           adicionar a instrução "jump whileX" à pilha de instruções, em que X=$1
           adicionar a label "whileX :" à pilha de instruções, em que X=$1+1
       }
     '}'
      ;
 
//--- EXPRESSAO ---//
Exp : NOT Exp { adicionar a instrução "pushi 0" à pilha de instruções
            adicionar a instrução "equal" à pilha de instruções }
    | Exp OPLT Exp { adicionar a instrução "sup" à pilha de instruções }
    | Exp OPGT Exp { adicionar a instrução "pushgp" à pilha de instruções }
    | Exp OPLE Exp { adicionar a instrução "infeq" à pilha de instruções }
    | Exp OPGE Exp { adicionar a instrução "supeq" à pilha de instruções }
    | Exp OPEQ Exp { adicionar a instrução "equal" à pilha de instruções }
    | Exp OPNE Exp { adicionar a instrução "pushi 0" à pilha de instruções
            adicionar a instrução "equal" à pilha de instruções }
    | Exp OPAND Exp { adicionar a instrução "mul" à pilha de instruções    }
    | Exp OPOR Exp { adicionar a instrução "add" à pilha de instruções }
    | Exp OPARITMAIS Exp { adicionar a instrução "add" à pilha de instruções }
    | Exp OPARITMENOS Exp { adicionar a instrução "sub" à pilha de instruções }
    | Exp OPARITPROD Exp { adicionar a instrução "mul" à pilha de instruções }
    | Exp OPARITDIV Exp { adicionar a instrução "div" à pilha de instruções }
    | Exp OPARITMOD Exp { adicionar a instrução "mod" à pilha de instruções }
    | VAR {
       se(não existe a variável em $1 || a variavel é do tipo array)
         erro
       senao
         adicionar a instrução "pushg X" à pilha de instruções, em que X=endereço($1)
     }
    | VAR {    
       se(não existe a variável em $1 || a variável é do tipo inteiro)
         erro
       senao
         adicionar a instrução "pushgp" à pilha de instruções
         adicionar a instrução "pushi X" à pilha de instruções, em que X=endereço($1)
         adicionar a instrução "padd" à pilha de instruções
     }
    '[' Exp ']'{
       adicionar a instrução "loadn" à pilha de instruções
     }
 
    | INT { adicionar a instrução "pushs X" à pilha de instruções, em que X=$1    }
    | '(' Exp ')'
    ;
\end{verbatim}

\section{Main do programa}

Aquando da execução do programa, depois de validar se existem argumentos, executam-se os seguintes passos:

\begin{itemize}
\item redirecionar o stdin do analisador sintático para um descritor do ficheiro fornecido como argumento;
\item executar a função yyparse (analisador sintático). Durante o decorrer desta função, a variável listaInstrucoes vai ser preenchida para posterior apresentação;
\item abrir para escrita um ficheiro com o mesmo nome do argumento fornecido, mas com a extensão .vm;
\item Verificar se existiram erros de compilação e:
\begin{itemize}
\item em caso afirmativo, escrever para o descritor do ficheiro .vm uma mensagem de erro;
\item em caso negativo, imprime cada uma uma das instruções na lista de instruções, separadas por um “\n”;
\end{itemize}
\end{itemize}
O código associado a este procedimento é apresentado em seguida.
\begin{verbatim}
int main(int argc, char *argv[]){
    if(argc < 2){
     printf("Introduzir ficheiro para compilação como argumento!\n");
     exit(0);
    }
    //--- PARSE ---//
    yyin = fopen(argv[1],"r");
     yyparse();
   //--- IMPRIMR OUTPUT ---//
   filename = strdup(argv[1]);
    filename[strrchr(filename,'.')-filename] = '\0';
    strcat(filename,".vm");
    FILE* f = fopen(filename,"w");
    if(!erros){
     while(listaInstrucoes){
       if(listaInstrucoes->instrucao){
         fprintf(f,"%s\n",listaInstrucoes->instrucao);
       }
       listaInstrucoes = listaInstrucoes->next;
     }
    }
    else {
     printf("Houve erros durante a compilação\n");
    }
    fclose(f);
     return 0;
}
\end{verbatim}


\section{Estruturas de Dados}
Foi necessário criar estruturas de dados capazes de albergar a informação para a correta execução dos programas. As estruturas criadas foram as seguintes:
 
\begin{verbatim}
struct variavel
{
  char *nome;
  char *tipo; 
  int endereco;
  struct variavel *next;
};
\end{verbatim}
Estrutura que armazena toda a informação respeitante à declaração de uma variável. O \texttt{nome} é o nome da variável, o \texttt{tipo} é o tipo da variável (ex: INT), o \texttt{endereco} é o endereço onde a variável está na VM e o \texttt{next} é o apontador para a próxima variável. Trata-se de uma lista ligada de variáveis.
 
\begin{verbatim}
struct instrucao
{
  int endereco;
  char *instrucao;
  struct instrucao *next;
};
\end{verbatim}
Estrutura que armazena uma instrução. O  \texttt{endereco} é um endereço no qual a instrução está localizada na VM, a  \texttt{instrucao} é a instrução em causa e  o  \texttt{next} é, à semelhança da declaração de variáveis, o apontador para a próxima instrução. Trata-se de uma lista ligada.
 
\begin{verbatim}
struct ifAddr
{
  int jz;
  struct ifAddr *next;
};
\end{verbatim}
Estrutura respeitante a uma condição If. A variável \texttt{jz} na estrutura armazena o salto condicional e a variável \texttt{next} armazena o endereço da próxima estrutura na lista ligada.
 
\begin{verbatim}
struct whileAddr
{
  int jump;
  int jz;
  struct whileAddr *next;
};
\end{verbatim}

Estrutura respeitante a um ciclo while. A variável  \texttt{jump} armazena o salto do início do while, a variável  \texttt{jz} armazena o salto condicional e a variável  \texttt{next} armazena o endereço da próxima estrutura na lista ligada.
\newline
\newline
Estas são as estruturas principais, para além destas foram também criadas outras estruturas mas que fazem uso destas.
 
\begin{verbatim}
struct variavel* insertVariavel(struct variavel *variavel, struct variavel *listaVariaveis);
int existeVariavel(char *variavel, struct variavel *listaVariaveis);
int enderecoVariavel(char *variavel, struct variavel *listaVariaveis);
char* tipoVariavel(char *variavel, struct variavel *listaVariaveis);
 
struct instrucao* insertInstrucao(int endereco, char *instrucao, struct instrucao *listaInstrucoes);
 
struct ifAddr* pushIfAddr(int endereco, struct ifAddr *pilhaIfAddr);
struct ifAddr* popIfAddr(struct ifAddr *pilhaIfAddr);
 
struct whileAddr* pushWhileAddr(int endereco, int whileAddr, struct whileAddr *pilhaWhileAddr);
struct whileAddr* popWhileAddr(struct whileAddr *pilhaWhileAddr);
 
void ifJump(int endereco, struct ifAddr *pilhaIfAddr, struct instrucao *pilhaInstrucoes);
void elseJump(int endereco, struct ifAddr *pilhaIfAddr, struct instrucao *pilhaInstrucoes);
int whileJump(int endereco, struct whileAddr *pilhaWhileAddr, struct instrucao *pilhaInstrucoes);
\end{verbatim}

\chapter{Programas Exemplo} \label{ae}

\section{Ordenação descendente - Bubble Sort}

\underline{Descrição:} programa que lê 10 números naturais e os escreve por ordem descendente.
\newline
\newline
\underline{Código-Fonte:}

\begin{verbatim}
i;
a[10];    
troca;    
trocado;  
&
enquanto(i<10){
    ler a[i];
    i++;
}
trocado=1;
enquanto(trocado){
    trocado=0;
    i=0;
    enquanto(i<10 -1 ){
     se(a[i]<a[i+1]){
       troca=a[i];
       a[i]=a[i+1];
       a[i+1]=troca;
       trocado=1;
     }
     i++;
    }
}
i=0;
enquanto(i<10){
    escrever a[i];
    escrever ",";
    i++;
}
\end{verbatim}

\underline{Assembly gerado:}

\begin{Verbatim}
pushi 0
     pushn 10
     pushi 0
     pushi 0
start
while0:
     pushg 0
     pushi 10
     inf
     jz while1
     pushgp
     pushi 1
     padd
     pushg 0
     read
     atoi
     storen
     pushg 0
     pushi 1
     add
     storeg 0
     jump while0
while1:
     pushi 1
     storeg 12
while2:
     pushg 12
     jz while3
     pushi 0
     storeg 12
     pushi 0
     storeg 0
while4:
     pushg 0
     pushi 10
     pushi 1
     sub
     inf
     jz while5
     pushgp
     pushi 1
     padd
     pushg 0
     loadn
     pushgp
     pushi 1
     padd
     pushg 0
     pushi 1
     add
     loadn
     inf
     jz if0
     pushgp
     pushi 1
     padd
     pushg 0
     loadn
     storeg 11
     pushgp
     pushi 1
     padd
     pushg 0
     pushgp
     pushi 1
     padd
     pushg 0
     pushi 1
     add
     loadn
     storen
     pushgp
     pushi 1
     padd
     pushg 0
     pushi 1
     add
     pushg 11
     storen
     pushi 1
     storeg 12
if0:
     pushg 0
     pushi 1
     add
     storeg 0
     jump while4
while5:
     jump while2
while3:
     pushi 0
     storeg 0
while6:
     pushg 0
     pushi 10
     inf
     jz while7
     pushgp
     pushi 1
     padd
     pushg 0
     loadn
     writei
     pushs ",";
     writes
     pushg 0
     pushi 1
     add
     storeg 0
     jump while6
while7:
Stop
\end{Verbatim}

\includegraphics[width=\textwidth]{screen1}

\section{Séries de Fibbonacci}

\underline{Descrição:} programa que lê um número natural e escreve a sua sequência de fibbonacci
\newline
\newline
\underline{Código-Fonte:}

\begin{verbatim}
i;
j;
k;
t;
n;
&
ler n;
j+=1;
escrever j;
k++;
enquanto(k<=n){
    t = i + j;
  i = j;
  j = t;
    k++;
    escrever j;
    escrever " , ";
}

\end{verbatim}

\underline{Assembly gerado:}

\begin{verbatim}
pushi 0
     pushi 0
     pushi 0
     pushi 0
     pushi 0
start
     read
     atoi
 storeg 4
     pushg 1
     pushi 1
     add
     storeg 1
     pushg 1
     writei
     pushg 2
     pushi 1
     add
     storeg 2
while0:
     pushg 2
     pushg 4
     infeq
     jz while1
     pushg 0
     pushg 1
     add
     storeg 3
     pushg 1
     storeg 0
     pushg 3
     storeg 1
     pushg 2
     pushi 1
     add
     storeg 2
     pushg 1
     writei
     pushs " , ";
     writes
     jump while0
while1:
stop
\end{verbatim}

\section{Números Ímpares}

\underline{Descrição:} programa que lê uma sequência de números terminada por 0 e imprime a contagem e os números ímpares da sequência.
\newline
\newline
\underline{Código-Fonte:}

\begin{verbatim}
numero;
conta;
&
numero=1;
escrever "IMPARES: ";
enquanto(numero){
    ler numero;
    se(numero%2){
     escrever(numero);
     escrever " , ";
     conta++;
    }
}
escrever "\nCONTAGEM:";
escrever conta;
\end{verbatim}

\underline{Assembly gerado:}

\begin{verbatim}
pushi 0
     pushi 0
start
     pushi 1
     storeg 0
     pushs "IMPARES: ";
     writes
while0:
     pushg 0
     jz while1
     read
     atoi
     storeg 0
     pushg 0
     pushi 2
     mod
     jz if0
     pushg 0
     writei
     pushs " , ";
     writes
     pushg 1
     pushi 1
     add
     storeg 1
if0:
     jump while0
while1:
     pushs "\nCONTAGEM:";
     writes
     pushg 1
     writei
stop
\end{verbatim}

\section{Ordem Inversa}

\underline{Descrição:} programa que lê uma sequência de 5 numeros naturais e imprime-os na ordem inversa.
\newline
\newline
\underline{Código-Fonte:}

\begin{verbatim}
n;
a[5];
i;
&
n=5;
enquanto(i<n){
    ler a[i];
    i++;
}
i=n-1;
enquanto(i>=0){
    escrever a[i];
    ESCREVER " , ";
    i--;
}
\end{verbatim}

\underline{Assembly gerado:}

\begin{verbatim}
pushi 0
     pushn 5
     pushi 0
start
     pushi 5
     storeg 0
while0:
     pushg 6
     pushg 0
     inf
     jz while1
     pushgp
     pushi 1
     padd
     pushg 6
     read
     atoi
     storen
     pushg 6
     pushi 1
     add
     storeg 6
     jump while0
while1:
     pushg 0
     pushi 1
     sub
     storeg 6
while2:
     pushg 6
     pushi 0
     supeq
     jz while3
     pushgp
     pushi 1
     padd
     pushg 6
     loadn
     writei
     pushs " , ";
     writes
     pushg 6
     pushi 1
     sub
     storeg 6
     jump while2
while3:
stop
\end{verbatim}

\section{Se’s aninhados}
\underline{Descrição:} programa de teste de vários se's aninhados (com e sem senao's)
\newline
\newline
\underline{Código-Fonte:}

\begin{verbatim}
a;
i;
&
se(a==0){
    escrever "SIM,";
    se(a==1){
     escrever "SIM,";
    }
    senao{
     enquanto(i<2){
       escrever "NAO,";
       se(a==1){
         escrever "SIM,";
         se(a==1){
           escrever "SIM,";
         }
         senao{
           escrever "NAO,";
         }
       }
       senao{
         escrever "NAO,";
         se(a==0){
           escrever "SIM,";
         }
         senao{
           escrever "NAO,";
         }
 
       }
       i+=1;
     }
    }
}
senao{
    escrever "NAO,";
    se(a==0){
     escrever "SIM,";
    }
    senao{
     escrever "NAO,";
    }
 
}
 
escrever " FIM";
\end{verbatim}

\underline{Assembly gerado:}

\begin{verbatim}
pushi 0
     pushi 0
start
     pushg 0
     pushi 0
     equal
     jz if4
     pushs "SIM,";
     writes
     pushg 0
     pushi 1
     equal
     jz if0
     pushs "SIM,";
     writes
     jump else0
if0:
while0:
     pushg 1
     pushi 2
     inf
     jz while1
     pushs "NAO,";
     writes
     pushg 0
     pushi 1
     equal
     jz if2
     pushs "SIM,";
     writes
     pushg 0
     pushi 1
     equal
     jz if1
     pushs "SIM,";
     writes
     jump else0
if1:
     pushs "NAO,";
     writes
else0:
     jump else1
if2:
     pushs "NAO,";
     writes
     pushg 0
     pushi 0
     equal
     jz if3
     pushs "SIM,";
     writes
     jump else1
if3:
     pushs "NAO,";
     writes
else1:
else2:
     pushg 1
     pushi 1
     add
     storeg 1
     jump while0
while1:
else3:
     jump else4
if4:
     pushs "NAO,";
     writes
     pushg 0
     pushi 0
     equal
     jz if5
     pushs "SIM,";
     writes
     jump else4
if5:
     pushs "NAO,";
     writes
else4:
else5:
     pushs " FIM";
     writes
stop
\end{verbatim}

\section{Lógica Clássica}

\underline{Descrição:} programa para testar operações lógicas e relacionais.
\newline
\newline
\underline{Código-Fonte:}

\begin{verbatim}
i;
f;
&
enquanto(i<10){
    se(!f){
     escrever " SIM ,";
    }
    senao{
     escrever " NAO ,";
    }
    se( f==1 ||  i%2==0 && i==5  ){
     escrever ",,JACKPOT,,";
    }
    f=(f+1)%2;
    i++;
}
\end{verbatim}

\underline{Assembly gerado:}

\begin{verbatim}
pushi 0
     pushi 0
start
while0:
     pushg 0
     pushi 10
     inf
     jz while1
     pushg 1
     pushi 0
     equal
     jz if0
     pushs " SIM ,";
     writes
     jump else0
if0:
     pushs " NAO ,";
     writes
else0:
     pushg 1
     pushi 1
     equal
     pushg 0
     pushi 2
     mod
     pushi 0
     equal
     add
     pushg 0
     pushi 5
     equal
     mul
     jz if1
     pushs ",,JACKPOT,,";
     writes
if1:
     pushg 1
     pushi 1
     add
     pushi 2
     mod
     storeg 1
     pushg 0
     pushi 1
     add
     storeg 0
     jump while0
while1:
stop
\end{verbatim}

\section{Menor número}
\underline{Descrição:} programa que lê n números (n fornecido pelo utilizador) e escreve o menor deles.
\newline
\newline
\underline{Código-Fonte:}

\begin{verbatim}
n;
aux;
menor;
&
ler n;
ler aux;
menor=aux;
n--;
enquanto(n){
    ler aux;
    se(aux<menor){
     menor=aux;
    }
    n--;
}
escrever menor;
\end{verbatim}

\underline{Assembly gerado:}

\begin{verbatim}
pushi 0
     pushi 0
     pushi 0
start
     read
     atoi
     storeg 0
     read
     atoi
     storeg 1
     pushg 1
     storeg 2
     pushg 0
     pushi 1
     sub
     storeg 0
while0:
     pushg 0
     jz while1
     read
     atoi
     storeg 1
     pushg 1
     pushg 2
     inf
     jz if0
     pushg 1
     storeg 2
if0:
     pushg 0
     pushi 1
     sub
     storeg 0
     jump while0
while1:
     pushg 2
     writei
stop
\end{verbatim}

\section{Operações Ariteméticas}

\underline{Descrição:} programa para testar o funcionamento das operações aritméticas e atribuições/leituras dos índices de um array.
\newline
\newline
\underline{Código-Fonte:}

\begin{verbatim}
b;
a[5];
&
a[0]=7/7;
a[1]= 5%3;
a[2]= 6-3;
a[3]= 2*2;
a[4]= 10;
escrever (b+1)*100/2;
escrever a[0];
escrever a[1];
escrever a[2];
escrever a[3];
escrever (a[4]);
\end{verbatim}

\underline{Assembly gerado:}

\begin{verbatim}
pushi 0
     pushn 5
start
     pushgp
     pushi 1
     padd
     pushi 0
     pushi 7
     pushi 7
     div
     storen
     pushgp
     pushi 1
     padd
     pushi 1
     pushi 5
     pushi 3
     mod
     storen
     pushgp
     pushi 1
     padd
     pushi 2
     pushi 6
     pushi 3
     sub
     storen
     pushgp
     pushi 1
     padd
     pushi 3
     pushi 2
     pushi 2
     mul
     storen
     pushgp
     pushi 1
     padd
     pushi 4
     pushi 10
     storen
     pushg 0
     pushi 1
     add
     pushi 100
     mul
     pushi 2
     div
     writei
     pushgp
     pushi 1
     padd
     pushi 0
     loadn
     writei
     pushgp
     pushi 1
     padd
     pushi 1
     loadn
     writei
     pushgp
     pushi 1
     padd
     pushi 2
     loadn
     writei
     pushgp
     pushi 1
     padd
     pushi 3
     loadn
     writei
     pushgp
     pushi 1
     padd
     pushi 4
     loadn
     writei
stop
\end{verbatim}

\section{Produtório}

\underline{Descrição:} programa que lê 5 números naturais e imprime o seu produtório.
\newline
\newline
\underline{Código-Fonte:}

\begin{verbatim}
n;
produtorio;
numero;
&
n=5;
produtorio=1;
enquanto(n){
    ler numero;
    produtorio=produtorio*numero;
    n--;
}
escrever produtorio;
\end{verbatim}

\underline{Assembly gerado:}

\begin{verbatim}
      pushi 0
     pushi 0
     pushi 0
start
     pushi 5
     storeg 0
     pushi 1
     storeg 1
while0:
     pushg 0
     jz while1
     read
     atoi
storeg 2
     pushg 1
     pushg 2
     mul
     storeg 1
     pushg 0
     pushi 1
     sub
     storeg 0
     jump while0
while1:
     pushg 1
     writei
stop
\end{verbatim}

\section{Quadrado}

\underline{Descrição:} programa que lê 4 números e verifica se podem corresponder aos lados de um quadrado.
\newline
\newline
\underline{Código-Fonte:}

\begin{verbatim}
a;
b;
c;
d;
&
ler a;
ler b;
ler c;
ler d;
se(a==b && b==c && c==d){
    escrever "É um quadrado.\n";
}
senao{
    escrever "Não é um quadrado.\n";
}
\end{verbatim}

\underline{Assembly gerado:}

\begin{verbatim}
pushi 0
     pushi 0
     pushi 0
     pushi 0
start
     read
     atoi
     storeg 0
     read
     atoi
     storeg 1
     read
     atoi
     storeg 2
     read
     atoi
     storeg 3
     pushg 0
     pushg 1
     equal
     pushg 1
     pushg 2
     equal
     mul
     pushg 2
     pushg 3
     equal
     mul
     jz if0
     pushs "É um quadrado.\n";
     writes
     jump else0
if0:
     pushs "Não é um quadrado.\n";
     writes
else0:
stop
\end{verbatim}

\section{Números Ímpares}

\underline{Descrição:} programa que lê 5 números naturais e imprime o seu somatório.
\newline
\newline
\underline{Código-Fonte:}

\begin{verbatim}
a;
num;
soma;
&
enquanto(a<5){
    ler num;
    soma+=num;
    a++;
}
escrever "Soma: ";
escrever soma;
\end{verbatim}

\underline{Assembly gerado:}

\begin{verbatim}
pushi 0
     pushi 0
     pushi 0
start
while0:
     pushg 0
     pushi 5
     inf
     jz while1
     read
     atoi
     storeg 1
     pushg 2
     pushg 1
     add
     storeg 2
     pushg 0
     pushi 1
     add
     storeg 0
     jump while0
while1:
     pushs "Soma: ";
     writes
     pushg 2
     writei
stop
\end{verbatim}


\chapter{Conclusão} \label{concl}
No geral, os objetivos esperados para o programa foram alcançados e obtivemos uma solução sólida para o problema apresentado, cumprindo o definido inicialmente.
Destacamos ainda alguns pontos:
\begin{itemize}
\item a GIC desenvolvida é de simples utilização, intuitiva e bastante completa;
\item foram implementadas várias formas de manuseamento de variáveis do tipo inteiro tradicionais: i++, i+=int,  arr[(i-int)*int ], entre outras.
\end{itemize}
\newline
Como trabalho futuro sugerimos:
\begin{itemize}
\item implementação de manipulação de arrays em 2 dimensões;
\item definição e invocação de subprogramas sem parâmetros que possam retornar um resultado atómico.
\end{itemize}

\bibliographystyle{alpha}
\bibliography{relprojLayout}

\end{document}