\chapter{Modelos Emocionais}

\section{Modelo emocional PAC}

O modelo de estados emocionais PAD procura descrever e medir os diferentes estados emocionais de uma entidade, com base em três dimensões: Satisfação (\textit{Pleasure}), Excitação (\textit{Arousal}) e Controlo (\textit{Dominance}). 

Estas dimensões são quantificadas numericamente, sendo que o nível da emoção é tanto maior quanto maior o seu valor. 
Além desta técnica de medição e representação simples, este sistema emocional assenta em conceitos que definem que o ambiente circundante afeta de forma direta o estado psicológico e emocional do individuo. 

Esta característica torna o modelo PAC viável de adaptar para um contexto de programação de robôs, dado que além de quantificar o estado emocional com apenas três níveis concretos, atribui a alteração do estado do agente a eventos e influências do ambiente. 
Neste sentido, apenas agentes com capacidade sensorial do ambiente podem implementar uma arquitetura baseada neste sistema, pois só assim serão capazes de percecionar os acontecimentos do meio onde se inserem. 

Na sua vertente psicológica, estas emoções podem ser descritas da seguinte forma: 

\begin{itemize}
    \item \textbf{Satisfação} (\textit{Pleasure}): Mede o quão satisfeito ou insatisfeito a entidade se encontra face a uma determinada situação. 
    
    Na perspetiva das emoções humanas, sentimentos como medo ou raiva levam a reduzidos níveis de satisfação. Em contrapartida, felicidade ou alegria levam a valores altos de satisfação. 
    
    \item \textbf{Excitação} (\textit{Arousal}): Mede a quantidade de energia da entidade em estudo. 
    
    A nível do comportamento humano, altos níveis de excitação podem estar, por exemplo, associados a situação de medo ou raiva, onde a quantidade de adrenalina aumenta na corrente sanguínea. Contextos enfadonhos ou cansaço são medidos com baixo nível de excitação. 
    
    \item \textbf{Controlo} (\textit{Dominance}): Representa o grau de controlo da entidade face uma determinada situação. O seu valor define se um sujeito apresenta  poder de atuação e decisão ou se age como um sujeito submisso. 
\end{itemize}

A nível do contexto de programação de \textit{robôs}, alguns dos seus comportamentos podem ser enquadrados nestas três dimensões. 
Conforme o nível de uma determinada dimensão, é possível alterar o estado interno do \textit{robô}, levando-o a tomar decisões distintas conforme o seu estado emocional. 

A titulo de exemplo, a listagem abaixo agrupa alguns dos comportamentos do robôs à respetiva dimensão emocional. 

\begin{enumerate}

    \item Se o robô destruir ou acertar num inimigo, o seu nível de satisfação e excitação sobem, sendo que destruir um adversário alterar o estado emocional de forma mais positiva que apenas acertar com um tiro;
    
    \item Se o robô for atingido com um tiro inimigo, o seu nível de satisfação e excitação descem;
    
    \item Se o robô representar o chefe da equipa, o seu nível de controlo é máximo, podendo com isso dar indicações a outros robôs da equipa;
    
    \item No inicio de cada ronda, os níveis de excitação devem ser máximos para cada robô, até porque o seu nível de energia é máxima;
    
    \item Robôs das classes mais simples, como o caso dos \textit{Droids}, devem apresentar reduzidos valores de controlo;
    
    \item Sempre que um robô se cruzar com um elemento da sua equipa, a sua excitação deve aumentar, simulando assim o incentivo trocado por dois elementos de equipa quando se cruzam; 
    
    \item De forma semelhante ao ponto anterior, sempre que um robô deteta no seu radar um inimigo, deve aumentar o seu nível de excitação, simulando assim a "adrenalina" libertada quando um inimigo se aproxima; 
    
    \item .... ToDo

\end{enumerate}

Este sistema emocional pode ser integrado no paradigma de uma arquitetura reativa, dado que a deteção de um evento e a determinação da sua consequência no estado emocional do agente, podem ser implementados com base em expressões condicionais pré estabelecidas. 


\section{Modelo emocional OCC}

O modelo emocional OCC, de forma semelhante ao modelo PAC, procura quantificar o peso de uma determinada emoção com base nos eventos, agentes ou objetos que pertencem ao ambiente no qual o agente que exibe a emoção se insere. 

Contudo, este modelo permite a representação de um número muito maior de emoções, baseando-se num conjunto de 5 fases para identificar e definir o estado inicial do agente. 
Apesar da sua extensão, as fases do processo de quantificação de emoções podem ser descritas da seguinte forma:

\begin{itemize}
    \item \textit{Classificação}: Fase onde o individuo avalia um evento, ação ou objeto, resultando num conjunto de novas informações que vão afetar diferentes categorias emocionais. 
    
    A nível da programação de robôs, esta fase corresponde à sensorização do meio, levando à aquisição de novas informações quando este muda de estado;
    
    \item \textbf{Quantificação}: Fase onde é quantificada a intensidade na qual cada categoria emocional é afetada. 
    
    Assim, para um determinado evento, é estimado o peso com que esse estimulo irá influenciar as diferentes traços do estado emocional do agente; 
    
    \item \textbf{Interação}: Determinada a influência que um evento irá causar nas categorias emocionais, esta fase realiza a alteração do estado dessas mesmas categorias, conforme o peso calculado na fase anterior; 
    
    \item \textbf{Mapeamento}: Alterado o estado emocional do agente, esta fase determina o conjunto de expressões que podem ser tomadas, com base no novo estado emocional. 
    
    A nível dos robôs, esta fase pode ser vista como o levantamento do conjunto de ações que podem ser tomadas, com base nos objetivos do robô e no estimulo recebido;
    
    \item \textbf{Expressão}: Por fim, a fase da expressão simboliza a execução de uma das decisões agrupadas e avaliadas na fase anterior. 
    
    Considerando o comportamento dos robôs, alguns eventos podem alterar o seu estado emocional mas não levar à execução imediata de uma ação. 
    Pode assim haver situações em que o robô não manifesta/executa ações, usando apenas a informação adquirira para alterar o seu estado interno. 
    
\end{itemize}

Dado que as tarefas associadas às diferentes fases deste modelo emocional são, de certa forma, subjetivas, a sua implementação enquadra-se numa arquitetura deliberativa. 
Integrando este modelo emocional complexo nas etapas de decisão deste tipo de arquiteturas, é assim possível modelar o comportamento e interpretação de eventos por parte de um robô, para um conjunto arbitrário de emoções e ações. 


\section{Modelo emocional OCEAN}

O modelo OCEAN, também conhecido como \textit{"Big Five"}, procura reduzir os principais traços emotivos num conjunto de 5 personalidades base, justificando assim a sua designação. 

Este conjunto de 5 classes, procura ser independente de qualquer fator cultural ou linguístico, agrupando os indivíduos em grupos que podem ser vistos como clusters para um determinado conjunto de comportamentos.

O estado emocional de um agente pode ser modelado através da atribuição de pesos a cada um destes grupos de personalidade, focando assim determinadas características de comportamento. 
Estes grupos podem ser descritos da seguinte forma:

\begin{itemize}
    \item \textbf{Curiosidade} (\textit{Openness}): Indivíduos com gosto em aprender e arriscar novas experiências, procurando conhecimento complexo, ambíguo e subtil, através da análise do ambiente onde se rodeiam.

    Associado a características humanas como cultura, originalidade e intelectualidade. 
    
    Robôs que se enquadrem neste grupo devem assumir um posição de vanguarda, arriscando executar ações novas e explorando o ambiente à sua volta, usando essa informação em ser proveito. 
    
    \item \textbf{Consciência} (\textit{Conscientiousness}): Caracteriza os elementos com vontade para conquistar e atingir um determinado objetivo. Os indivíduos demonstram assim uma disciplina interna, com capacidade de planeamento e escolha de comportamentos com base nos seus objetivos. 
    
    Robôs com este tipo de características devem ser capazes de tomar as suas decisões tendo em conta o plano de ação da equipa, atingindo assim os objetivos da mesma. 
    
    \item \textbf{Extraversão} (\textit{Extraversion}): Indivíduos com energia, capacidade de dialogo e assertividade. A sua energia e motivação é adquirira através da comunicação com outros elementos do ambiente. 
    
    Robôs neste segmento devem apresentar uma elevada capacidade de comunicação com os restantes elementos da sua equipa. 
    As suas ações e motivações são assim adaptadas conforme o conjunto de informações e decisões trocados com os restantes elementos da equipa. 
    
    \item \textbf{Conformismo} (\textit{Agreeableness}): Indivíduos com características passivas e amigáveis, capazes de se acomodar e cooperar com qualquer contexto que lhes seja imposto. Representam o oposto dos elementos competitivos e antagonistas de um determinado ambiente. 
    
    Robôs enquadrados neste grupo devem estar associados às classes mais simples de robôs, capazes de facilmente moldar o seu comportamento conforme as instruções que recebem de outros robôs da sua equipa. 
    Podem ser vistos como os elementos obedientes da equipa, com uma capacidade limitada para julgar ou contrariar as ações que lhes são incutidas. 
    
    \item \textbf{Neuroticismo} (\textit{Neuroticism}): Grupo de indivíduos que manifesta um conjuntos de emoções negativas, explorando este tipo de sentimentos. 
    Apresentam reações exageradas, baseadas em emoções negativas, para atuar perante um determinado evento. 
    
    Robôs com este tipo de comportamento compulsivo e negativo podem manifestar sentimentos de raiva ou vingança.
    Por exemplo, ao encontrarem um inimigo, perseguirem o mesmo na tentativa de o abater, deixando de seguir o plano de ação geral da equipa. 

    
\end{itemize}

