\section{Introdução}\label{sec:Introduction}

Neste projeto pretende-se fazer um estudo aprofundado sobre code smells, as respetivas técnicas de refactoring e a sua aplicabilidade. Como ponto de partida servem os trabalhos práticos da disciplina onde se pretendia desenvolver um Trader de CFDs. É assim esperado uma melhoria da estrutura interna após a aplicação das técnicas acima referidas.
Este relatório serve como descrição dos procedimentos realizados durante o projeto, sendo que este está dividido em três partes distintas: uma primeira parte onde se descrevem os procedimentos referentes ao trabalho prático número 1, uma segunda parte onde se descrevem os procedimentos referentes ao trabalho prático número 2 e por fim, selecionou-se um anti-padrão não presente no código original e aplicou-se este anti-padrão.
\vspace{5mm}
\newline As ferramentas usadas foram:

\begin{enumerate}
    \item SourceMonitor (métricas estáticas)
    \item Eclipse (IDE)
    \item TheadMXBEan Library
    \item nanoTime Library
\end{enumerate}
\vspace{5mm}
\textbf{Métricas dinâmicas}
\begin{lstlisting}[breaklines,frame=single,language=java]
long startTimeNano = System.nanoTime( );
ThreadMXBean threadMXBean = ManagementFactory.getThreadMXBean();
long time = threadMXBean.getCurrentThreadCpuTime();
long taskTimeNano  = System.nanoTime( ) - startTimeNano; 
\end{lstlisting}