\chapter{\emph{Project Issues}}

As seguintes secções contêm problemas que devem ser enfrentados se os requisitos são para serem alcançados e o se o produto é para se tornar uma realidade.
Estas secções também fazem a ligação dos requisitos com as atividades do projeto que fazem a descoberta e progresso dos requisitos. Se a linguagem usada é consistente para comunicar requisitos então os gestores de projeto podem usar os requisitos como input para conduzir o projeto.

\section{Temas Abertos}

Até ao momento não foram detetados problemas durante o processo de levantamento de requisitos.

\section{Soluções Disponíveis}

\subsection{Produtos Prontos}
Não foram detetados.

\subsection{Componentes Reutilizáveis}

Existem dois componentes já desenvolvidos pela \emph{F3M Information Systems S.A}. que irão ser integrados nesta aplicação. Está já desenvolvida uma plataforma de \emph{scouting} de jogadores que irá servir de ponto de partida para o desenvolvimento, bem com a base de dados que a suporta. 

\subsection{Produtos que podem ser copiados}

A aplicação do calendário de cada \emph{smartphone} permite criar eventos e criar alertas para qualquer coisa que o utilizador introduza. O funcionamento de introdução de uma nova ocorrência é semelhante, mas não é algo que possa ser copiado por inteiro.

\section{Novos Problemas}

\subsection{Efeitos sobre o Ambiente Atual}

A implementação deste produto irá ter alguns efeitos sobre o ambiente para o qual se destina. Começando pela diminuição de casos de não pagamento de direitos de formação, solidariedade ou direitos de transferência futura. Uma transferência que poderia passar como desconhecida para um clube que tenha direito a receber dinheiro pode agora ser do conhecimento de todos.
No caso dos agentes e dos treinadores, este produto auxilia o acompanhamento da vida de pessoal dos jogadores, seja para os motivar ou apoiar em momentos difíceis, situações que poderiam passar em claro sem a implementação deste produto. Como consequência, o jogador sente-se mais acompanhado e com menos preocupações para se focar apenas na sua performance.

\subsection{Problemas Potenciais do Utilizador}

Alguns utilizadores poderão usar a aplicação como único suporte e descriminar o contacto pessoal, isto no caso dos agentes e treinadores. Outro fator importante é a introdução de informação falsa que poderá levar a decisões erradas.


\section{Tarefas}

\subsection{Planeamento do Projeto}

A elaboração deste documento foi feita com o intuito de futuramente fazer a implementação da aplicação, mas sem data definida, por isso, o planeamento do projeto não foi contemplado aqui.
Como o projeto é feito em parceria, todo o planeamento tem de ser feito com o aval e supervisão da \emph{F3M Information Systems, S.A}.

\subsection{Planeamento das Fases de Desenvolvimento}

Não identificadas.

\section{Migração para o Novo Produto}

Para avançar com a construção desta aplicação irá ser necessário ter em atenção ao modo como a base de dados está implementada. Neste momento não temos conhecimento técnico sobre a sua implementação e por conseguinte não é possível elaborar um lista de atividades de conversão necessárias antes de avançar com o projeto.

\section{Riscos}

Durante a elaboração do projeto foram identificados alguns riscos que poderão afetar o funcionamento esperado do produto:
\begin{itemize}
    \item \underline{\bf{Dados incorretos:}} Como a aplicação tem como base de informação uma base de dados externa há sempre o risco de haver dados incorretos ou incompletos que podem criar situações desagradáveis para o utilizador. Por exemplo: \emph{Um treinador deseja um feliz aniversário a um jogador num dia errado.}
    
    \item \underline{\bf{Segurança dos dados:}} Como a aplicação lida com dados sensíveis, tanto pessoais como contratuais é imperativo garantir a segurança dos mesmos, onde se inclui os pedidos à base de dados, altura em que o sistema se encontra vulnerável a roubo de informações. Para isso é recomendado implementar um sistema robusto e atualizar o conhecimento de novas ameaças.
 
\end{itemize}

\section{Custos}

Esta secção refere-se ao custo de implementar os requisitos definidos, mas como foi referido anteriormente, este projeto é feito em parceria com uma empresa e tudo o que é relacionado com implementação e custos ainda não foi abordado de forma a que se possa fazer uma estimativa.

\section{Documentação e Treino dos Utilizadores}

Será necessário produzir documentação a explicar as funcionalidades da aplicação, no entanto, não se trata de algo aprofundado em termos técnicos. Os utilizadores alvo da aplicação estão caracterizados como utilizadores de \emph{smartphones} e computadores, por isso é expectável que o esforço para familiarização com a plataforma não seja muito. Esta documentação pode ainda ser convertida num tutorial dentro da aplicação à semelhança da prática atual.

\section{Sala de Espera}

Requisitos que poderão ser incluídos em futuras implementações do sistema são os identificados com a prioridade \emph{\bf{W}} nas secções de requisitos funcionais e não funcionais.

\section{Ideias para Soluções}

Não foram identificadas novas ideias para o caso em questão.