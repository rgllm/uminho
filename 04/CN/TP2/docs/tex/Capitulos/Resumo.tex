\begin{abstract}

Com o massivo volume de publicações que diariamente se realizam nas diversas redes sociais, torna-se relevante realizar uma análise de alto nível do seu conteúdo.
Aliando esta fonte de dados à atual capacidade de processamento que se consegue obter em métodos de \textit{Machine Learning} é assim viável a exploração destes dados com base nestas técnicas, como forma de análise dos mesmos. 

Neste sentido, o presente projeto prático pretende explorar a utilização de \textit{Redes Neuronais Artificiais} (\textit{RNAs}) num problema de classificação emocional de texto, através de publicações na rede social \textit{Twitter}. 

Ao longo deste relatório é apresentado todo o processo desenvolvido como forma de analisar o conjunto de dados em bruto, o pré processamento do mesmo, o desenvolvimento da arquitetura da RNA e o processo de treino e teste da mesma. 

Por fim, são apresentados os resultados obtidos, focando os padrões de treino que obtêm melhores resultados de classificação. 

\end{abstract}