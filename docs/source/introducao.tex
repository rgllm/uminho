\section{Introdução}\label{sec:Introduction}

O projeto proposto no âmbito da unidade curricular de Agentes Inteligentes tem como objetivo desenvolver um sistema no ambiente multiagente JADE, complementando com a utilização de JADEX e JESS.
O projeto consiste no desenvolvimento de um Sistema de Partilha de Bicicletas, que permita aos utilizadores alugar bicicletas numa dada estação, realizar uma viagem e entregar a bicicleta numa outra estação. No entanto, o grande problema em jogo é a capacidade das estações. Como as estações de recolha/entrega têm uma capacidade limitada torna-se importante fazer a gestão das bicicletas na cidade, não deixando uma estação ficar sem bicicletas e outra com excesso, impedindo a sua devolução. A isto se chama o Problema de Reequilíbrio de Partilha de Bicicletas (PRPB). Uma das formas de resolução deste problema é o incentivo a utilizadores, para que estes entreguem as bicicletas em estações menos lotadas. A devolução ou não das bicicletas por parte dos utilizadores é calculada através de diversos fatores que influenciam a escolha.