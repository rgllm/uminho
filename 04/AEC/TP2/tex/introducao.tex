\section{Introdução}\label{sec:Introduction}

O ritmo de recolha e armazenamento de dados a partir de contextos reais aumenta de forma dramática, dia após dia. Torna-se assim praticamente impossível tratar e explorar estes dados manualmente. Não só porque os dados recolhidos a partir do mundo real não podem ser imediatamente utilizados para a extração de conhecimento já que muitas vezes apresentam diversos problemas de incoerências, inconsistências ou até mesmo dados inválidos. Surge assim a necessidade de tratar os dados recolhidos e só posteriormente extrair o conhecimento possível, de forma digital.
Este tratamento segue uma série de passos que vão desde a seleção, o pré-processamento, a transformação, a mineração dos dados e posterior interpretação dos resultados.
Extração de Conhecimento é assim, o processo de descoberta de conhecimento a partir de dados de fontes.
O processo de mineração de dados é o mais importante já que consiste em aplicar algoritmos específicos para extrair padrões dos dados, contudo este passo na maioria das vezes é aquele que ocupa menos tempo. De entre os algoritmos para este processo destacam-se a Regressão, a Classificação, a Segmentação e a Associação.
\newline
Neste trabalho será realizado um processo de Extração de Conhecimento em dois conjuntos de dados distintos, com suporte da ferramenta WEKA.