\chapter{Aplicação do Método}
\label{cht:aplicacao}

Neste capítulo é aplicado a metodologia desenvolvida nos capítulos anteriores a dois projetos, como forma de demonstração e validação do que foi desenvolvido.
\\O primeiro projeto é um projeto desenvolvido no contexto académico e o segundo é um projeto de uma empresa.

\vspace{5mm}

\subsection{Projeto académico}

O projeto foi proposto aos alunos do Mestrado em Engenharia Informática na unidade curricular de Engenharia Web. Em grupos de no máximo 4 elementos, os alunos devem desenvolver com tecnologias Web uma plataforma de trading. O prazo de entrega é o dia 14 de junho, sendo que existe uma entrega intermédia dia 10 de maio, o que dá cerca de 4 meses para desenvolver todo o projeto. É necessário desenvolver tanto o frontend da aplicação como o backend, o projeto tem cerca de 8 requisitos obrigatórios.

\begin{framed}
\noindent\textbf{Metodologia de Diagnóstico de um Projeto Ágil}
\qquad
Versão 1.0
\vspace{2mm}
\newline Este método de diagnóstico deve ser usado na fase de pré-planeamento de um projeto e servirá como uma escolha inicial entre os métodos mais tradicionais e um método Ágil. A primeira versão deste questionário apenas serve para indicar se deve ou não ser usada uma Metodologia Ágil, o método Ágil em específico não é tido em conta.
\vspace{1mm}
\newline Este questionário é constituído por 12 perguntas. As perguntas podem ter cotação 1, 2 ou 3 o que mede a sua importância e serve como fator multiplicativo.
\vspace{5mm}
\newline\textbf{Questão 1 - } Há uma necessidade dentro da sua organização de mudar de método de desenvolvimento?
\newline \begin{center} Sim (1 ponto) \hspace{30mm} \textbf{Não (0 pontos)}\end{center}
\vspace{2mm}
\newline\textbf{Questão 2 - } O projeto tem urgência e os prazos para sua conclusão são apertados?
\newline \begin{center} \textbf{Sim (2 pontos)} \hspace{30mm} Não (0 pontos)\end{center}
\vspace{2mm}
\newline\textbf{Questão 3 - } A sua equipa é bastante organizada?
\newline \begin{center} Sim (1 ponto) \hspace{30mm} \textbf{Não (0 pontos)}\end{center}
\vspace{2mm}
\newline\textbf{Questão 4 - } O cliente necessita de documentação clara de cada ciclo de desenvolvimento?
\newline \begin{center} Sim (0 pontos) \hspace{30mm} \textbf{Não (2 pontos)}\end{center}
\vspace{2mm}
\newline\textbf{Questão 5 - } O cliente necessita de aprovar o projeto a cada fase de desenvolvimento?
\newline \begin{center} \textbf{Sim (0 pontos)} \hspace{30mm} Não (1 ponto)\end{center}
\vspace{2mm}
\newline\textbf{Questão 6 - } O cliente está mais habituado e prefere usar métodos de desenvolvimento mais tradicionais?
\newline \begin{center} Sim (0 pontos) \hspace{30mm} \textbf{Não (2 pontos)}\end{center}
\newline\textbf{Questão 7 - } Na sua perspetiva e mediante os projetos que a sua organização costuma trabalhar, qual é o tamanho do projeto?
\newline \begin{center} Pequeno (2 pontos) \hspace{17mm} \textbf{Médio (1 ponto)}\hspace{17mm}Grande (0 pontos)\end{center}
\vspace{2mm}
\newline\textbf{Questão 8 - } São necessárias múltiplas variantes do projeto, ou pelo menos, são desejáveis?
\newline \begin{center} Sim (1 ponto) \hspace{30mm} \textbf{Não (0 pontos)}\end{center}
\vspace{2mm}
\newline\textbf{Questão 9 - } Qual é a área do projeto?
\newline \begin{center} \textbf{Software ou Hardware (3 pontos)}\hspace{17mm} Serviços Financeiros (2 pontos)\hspace{17mm}Serviços Profissionais (1 ponto)\hspace{17mm} Outros (0 pontos)\end{center}
\vspace{2mm}
\newline\textbf{Questão 10 - } O projeto está bem documentado e todos os requisitos estão definidos à partida?
\newline \begin{center} \textbf{Sim (0 pontos)} \hspace{30mm} Não (2 pontos)\end{center}
\vspace{2mm}
\newline\textbf{Questão 11 - } O cliente terá uma participação ativa no desenvolvimento do produto?
\newline \begin{center} Sim (1 ponto) \hspace{30mm} \textbf{Não (0 pontos)}\end{center}
\vspace{2mm}
\newline\textbf{Questão 12 - } Os elementos da sua equipa são pró-ativos e demonstram iniciativa?
\newline \begin{center}\textbf{Sim (1 ponto)} \hspace{30mm} Não (0 pontos)\end{center}

\vspace{10mm}
\begin{center}
\begin{itemize}
    \item \textbf{Menos de 10 pontos} - Os Métodos Ágil \\não são, à partida, apropriados ao seu projeto.
    \item \textbf{Entre 10 e 15 pontos} - Os Métodos Ágil talvez \\sejam apropriados mas necessita de uma segunda opinião.
    \item \textbf{Mais de 15 pontos} - Os Métodos Ágil são\\ apropriados ao seu projeto.
\end{itemize}
\end{center}
\end{framed}

O total de pontos obtidos é \textbf{11 pontos}. Por isto, o projeto fica no patamar intermédio. É por isso necessário uma segunda opinião, ou pelo menos, uma reflexão mais profunda e a longo prazo se é vantajoso usar os Métodos Ágeis.

\newpage

\subsection{Projeto empresarial - Aplicação para supermercado}

O projeto a ser desenvolvido tem como finalidade o desenvolvimento de uma aplicação mobile para uma conhecida marca de supermercados. Esta aplicação deverá ser capaz de receber pedidos de compras dos clientes a partir de uma listagem de todos os produtos em loja. A  equipa de desenvolvimento vai ser constituída por 8 elementos, entre designers, direção e programadores. O prazo para a conclusão e entrega final da aplicação é o dia 21 de junho, o que desde o pedido inicial até à entrega final dá cerca de 6 meses de trabalho. O cliente apenas fez o pedido inicial, mas não definiu os requisitos totalmente, estes serão definidos à medida que o desenvolvimento vai acontecendo. É necessário uma aplicação para iOS e outra para Android.

\begin{framed}
\noindent\textbf{Metodologia de Diagnóstico de um Projeto Ágil}
\qquad
Versão 1.0
\vspace{2mm}
\newline Este método de diagnóstico deve ser usado na fase de pré-planeamento de um projeto e servirá como uma escolha inicial entre os métodos mais tradicionais e um método Ágil. A primeira versão deste questionário apenas serve para indicar se deve ou não ser usada uma Metodologia Ágil, o método Ágil em específico não é tido em conta.
\vspace{1mm}
\newline Este questionário é constituído por 12 perguntas. As perguntas podem ter cotação 1, 2 ou 3 o que mede a sua importância e serve como fator multiplicativo.
\vspace{5mm}
\newline\textbf{Questão 1 - } Há uma necessidade dentro da sua organização de mudar de método de desenvolvimento?
\newline \begin{center} Sim (1 ponto) \hspace{30mm} \textbf{Não (0 pontos)}\end{center}
\vspace{2mm}
\newline\textbf{Questão 2 - } O projeto tem urgência e os prazos para sua conclusão são apertados?
\newline \begin{center} \textbf{Sim (2 pontos)} \hspace{30mm} Não (0 pontos)\end{center}
\vspace{2mm}
\newline\textbf{Questão 3 - } A sua equipa é bastante organizada?
\newline \begin{center} \textbf{Sim (1 ponto)} \hspace{30mm} Não (0 pontos)\end{center}
\vspace{2mm}
\newline\textbf{Questão 4 - } O cliente necessita de documentação clara de cada ciclo de desenvolvimento?
\newline \begin{center} Sim (0 pontos) \hspace{30mm} \textbf{Não (2 pontos)}\end{center}
\vspace{2mm}
\newline\textbf{Questão 5 - } O cliente necessita de aprovar o projeto a cada fase de desenvolvimento?
\newline \begin{center} \textbf{Sim (0 pontos)} \hspace{30mm} Não (1 ponto)\end{center}
\vspace{2mm}
\newline\textbf{Questão 6 - } O cliente está mais habituado e prefere usar métodos de desenvolvimento mais tradicionais?
\newline \begin{center} Sim (0 pontos) \hspace{30mm} \textbf{Não (2 pontos)}\end{center}
\newline\textbf{Questão 7 - } Na sua perspetiva e mediante os projetos que a sua organização costuma trabalhar, qual é o tamanho do projeto?
\newline \begin{center} Pequeno (2 pontos) \hspace{17mm} \textbf{Médio (1 ponto)}\hspace{17mm}Grande (0 pontos)\end{center}
\vspace{2mm}
\newline\textbf{Questão 8 - } São necessárias múltiplas variantes do projeto, ou pelo menos, são desejáveis?
\newline \begin{center} \textbf{Sim (1 ponto)} \hspace{30mm} Não (0 pontos)\end{center}
\vspace{2mm}
\newline\textbf{Questão 9 - } Qual é a área do projeto?
\newline \begin{center} \textbf{Software ou Hardware (3 pontos)}\hspace{17mm} Serviços Financeiros (2 pontos)\hspace{17mm}Serviços Profissionais (1 ponto)\hspace{17mm} Outros (0 pontos)\end{center}
\vspace{2mm}
\newline\textbf{Questão 10 - } O projeto está bem documentado e todos os requisitos estão definidos à partida?
\newline \begin{center} Sim (0 pontos) \hspace{30mm} \textbf{Não (2 pontos)}\end{center}
\vspace{2mm}
\newline\textbf{Questão 11 - } O cliente terá uma participação ativa no desenvolvimento do produto?
\newline \begin{center} \textbf{Sim (1 ponto)} \hspace{30mm} Não (0 pontos)\end{center}
\vspace{2mm}
\newline\textbf{Questão 12 - } Os elementos da sua equipa são pró-ativos e demonstram iniciativa?
\newline \begin{center} \textbf{Sim (1 ponto)} \hspace{30mm} Não (0 pontos)\end{center}

\vspace{10mm}
\begin{center}
\begin{itemize}
    \item \textbf{Menos de 10 pontos} - Os Métodos Ágil \\não são, à partida, apropriados ao seu projeto.
    \item \textbf{Entre 10 e 15 pontos} - Os Métodos Ágil talvez \\sejam apropriados mas necessita de uma segunda opinião.
    \item \textbf{Mais de 15 pontos} - Os Métodos Ágil são\\ apropriados ao seu projeto.
\end{itemize}
\end{center}
\end{framed}

O total de pontos obtidos é \textbf{17 pontos}. Por isto, o projeto fica no patamar máximo. O projeto deverá sem dúvida usar uma metodologia Ágil já que se adapta às principais características deste tipo de método.