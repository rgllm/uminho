\chapter{Análise dos Resultados e Conclusão}
%\label{sec:Conclusion}

Como podemos ver pelos gráficos acima, a refabricação do código teve impacto no consumo energético. Como dito na secção \ref{refactor} a refabricação de código tem como objetivo reestruturar o código de forma a torna-lo mais compreensível por outros programadores e aumentar a sua reutilização e manutenção. Algumas técnicas de refabricação contribuem também para tornar o código mais eficiente, nomeadamente o exemplo fornecido no excerto de código em \ref{exemplo}, em que uma condição mais complexa é substituída por uma mais simples. No programa original era usado o método \texttt{size()} da collection e comparado com o valor $0$, que pode ser otimizado usando o método de classe mais apropriado para verificar se uma coleção tem $0$ elementos - \texttt{isEmpty()}. 
\newline
Outro mau cheiro que tem implicações negativas na performance e consequentemente no consumo energético é o \textit{middle man} - quando uma classe apenas redireciona os métodos para outras classes. Apesar de este caso não ter sido encontrado no projeto analisado, seria outro caso em que se esperaria que a sua refabricação tivesse um impacto positivo no consumo energético.
\newline
Assim, podemos concluir que a refabricação é um processo com muitas vantagens e que deve ser aplicada sempre que possível, principalmente quando queremos código mais reutilizável e compreensível, e também quando temos (que deverá ser sempre) preocupação em produzir Green Software.
\newline
Por tudo isto, consideramos que os objetivos do enunciado foram todos cumpridos. Na nossa opinião trabalhar com novas ferramentas como o Cobertura e o RAPL foi uma oportunidade de enorme valor, constituindo uma boa forma de aprender ferramentas muito úteis para o futuro.