\section*{Introdução}\label{sec:Introduction}

Inteligência Artificial, Redes Neuronais e Sistemas Inteligentes são cada vez mais termos presentes no nosso dia a dia. Habituamo-nos a ver estes conceitos por todo o lado, seja no nosso telemóvel novo, no nosso carro ou até mesmo no nosso novo frigorífico. As Redes Neuronais Artificiais têm um papel muito importante nesta expansão da Inteligência Artificial, já que permitiram resolver problemas como um modelo simplificado do sistema nervoso central dos seres humanos. Este tipo de sistema de base conexionista apresenta uma estrutura interligada de unidades computacionais, também designados por neurónios que apresentam capacidade de aprendizagem.
\\Pela sua importância e atualidade existem, cada vez mais, diversas soluções disponíveis para o desenvolvimento de Redes Neuronais Artificiais. Durante este estudo apresenta-se uma comparação de algumas das soluções atualmente disponíveis, indicando os seus pontos fracos e fortes e métodos de aprendizagem.