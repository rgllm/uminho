\chapter{Entrevista a um Treinador de Futebol}

\section{Preparação da Entrevista}

Na preparação desta entrevista o grupo decidiu juntar-se e criar um conjunto de perguntas relevantes para fazer ao treinador entrevistado.
Combinou-se um encontro com o treinador e pediu-se autorização ao entrevistado para utilizar um gravador de voz para facilitar a transcrição das respostas obtidas ao longo da entrevista.

\section{Entrevista a Daniel Magalhães, treinador da equipa de futebol Sanjoanense}

No início da entrevista, o entrevistado foi contextualizado do âmbito da mesma e do que se trata a aplicação que se pretende desenvolver. Depois de bem entendido o domínio da aplicação pela parte do entrevistado, procedeu-se à entrevista propriamente dita.
\vspace{5mm}
\newline\textbf{P: Que tipos de ocorrências sobre os jogadores é que um treinador tem interesse em saber?}
\vspace{2mm}
\newline\textbf{R:} Em termos coletivos, gostamos de saber o número de perdas de bola, número de recuperações de bola, número de interseções feitas, numero de passes, número de remates, número de esquemas táticos, entre outros fatores. Tudo isto dividido em setores (dois meios campos). Em termos individuais, saber o número de passes falhados, total de passes (a partir destes dois parâmetros obtém-se o número de passes bem sucedidos), a  distância percorrida, número de interseções... No final de cada jogo, saber qual foi o jogador com melhor rendimento.
\vspace{2mm}
\newline\textbf{P: O que interessa a um treinador saber sobre um jogador quando existe interesse em contrata-lo?}
\vspace{2mm}
\newline\textbf{R:} Como é óbvio, as suas capacidades técnicas e físicas são o principal foco, mas também é extremamente importante saber sobre o seu histórico de lesões, capacidade social, companheirismo, relacionamento com os outros, responsabilidade e satisfação com o clube onde está. Estes fatores são também decisivos na hora de decidir oferecer ou não um contrato a um jogador.
\vspace{2mm}
\newline\textbf{P: Como idealiza a melhor forma de consultar as ocorrências ao interagir com uma aplicação deste tipo?}
\vspace{2mm}
\newline\textbf{R:} Quem de dera a mim, ao chegar ao balneário no intervalo do jogo, abrir a aplicação num \emph{tablet} ou telemóvel e ter acesso à análise individual e coletiva da equipa. Essa informação permitiria uma grande melhoria nas decisões a tomar e instruções a dar aos jogadores para o arranque da segunda parte. E, como é óbvio, também no final do jogo estas informações são importantes para fazer um balanço do rendimento da equipa coletiva e individualmente.
\vspace{2mm}
\newline\textbf{P: Como idealiza a melhor forma de inserir as ocorrências ao interagir com uma aplicação deste tipo?}
\vspace{2mm}
\newline\textbf{R:} O treinador de adjunto pode durante o jogo tratar de registar as ocorrências na aplicação, ou então um sistema que vídeo que consiga fazer essa análise durante o jogo.
\vspace{2mm}
\newline\textbf{P: Conhece alguns serviços semelhantes a este que pretendemos desenvolver?}
\vspace{2mm}
\newline\textbf{R:} \emph{Instat},\emph{Transfer Market} e o site \emph{lpfp.pt} são algumas das ferramentas usadas pelos treinadores para analisar as estatísticas sobre os jogadores.
\vspace{2mm}
\newline\textbf{P: Como é que é feito atualmente o controlo das penalizações dos jogadores (impedimento de jogar por demasiados cartões, etc…)}
\vspace{2mm}
\newline\textbf{R:} A Federação Portuguesa de Futebol envia um email semanalmente aos clubes com todas as informações relevantes.
\vspace{3mm}
\newline
No final da entrevista foi mostrada ao entrevistado a lista com todas as ocorrências que foram por nós identificadas como sendo de interesse para um treinador, e foi-lhe pedido que avaliasse a relevância de cada uma delas. Os comentários feitos pelo entrevistado permitiram-nos concluir que todas as ocorrências que identificamos têm realmente interesse, umas mais que outras, e nenhuma delas foi considerada inútil. O entrevistado acrescentou ainda: “Todas estas coisas têm interesse para nós porque é realmente importante saber o máximo possível sobre a pessoa que potencialmente virá no futuro a trabalhar com a nossa equipa."
\vspace{2mm}

\section{Resumo da entrevista}
Com a análise da entrevista, apercebemo-nos que podia ter havido um melhor direcionamento da conversa, visto que o entrevistado respondeu a algumas perguntas com assuntos que não pertencem realmente do domínio da nossa aplicação. Mesmo tendo isso em conta, a entrevista ajudou-nos a recolher informações de elevada importância para o levantamento de requisitos.
Concluímos também que o entrevistado sente a necessidade da existência de uma aplicação deste tipo, acreditando que talvez as grandes equipas de futebol já tenham acesso a ferramentas semelhantes, e afirma que para equipas mais pequenas seria realmente útil ter acesso a tais ferramentas, mas a um preço que estas equipas consigam suportar.
